

\definecolor{tcA}{rgb}{0.9,0.9,0.9}

\newcommand\BackgroundPic{
	\put(0,0){
		\parbox[b][\paperheight]{\paperwidth}{%
			\vfill
			\centering
			\includegraphics[width=\paperwidth,height=\paperheight,
			keepaspectratio]{./img/logo6.png}%
			\vfill
}}}

\newcommand{\universidad}{\large{Universidad de Matanzas }}
\newcommand{\facultad}{\large{Facultad de Ciencias Sociales y Humanidades}}

\newcommand{\espacios}{\vspace{0.5in}}

\newcommand{\espaciosH}{\hspace{0.5in}}

\newcommand{\entidad}
{
 	\begin{tabular}{c}
 		
 		\universidad \\
 		\facultad \\ 
 		\espacios \\
 		
 	\end{tabular}
}

\newcommand{\logoUniversidad}{\includegraphics[scale=0.69]{img/logo_umcc.png}}
\newcommand{\logoFacultad}{\includegraphics[scale=0.03]{img/logo_facultad_fcsh_negro.png}}

\newcommand{\banner}
{
 \espacios
\begin{center}
\begin{tabular}{ccc}

\logoUniversidad  & \espaciosH \entidad  & \espaciosH \logoFacultad  \\

\end{tabular}
\end{center}


 \espacios
 }


% Escribir el titulo de la tesis que va a exponer.
\newcommand{\tituloTesis}{ INFORME DE AUTOEVALUACIÓN
\\ CARRERA: GESTIÓN SOCIOCULTURAL PARA EL DESARROLLO}

\newcommand{\tituloTesisPrimeraPagina}{ {\fontsize{17pt}{17pt}\selectfont \tituloTesis} }


% Cambiar el tipo de trabajo si es necesario.
\newcommand{\nombreTipoTesis}{ {\fontsize{14pt}{14pt}\selectfont Trabajo de Diploma  para la Obtención del Título de Ingeniero Informático}}

% Cambiar los nombres de los autores.
\newcommand{\autorUNO}[1]{{\fontsize{16pt}{16pt}\selectfont #1}}
\newcommand{\autorFIRMA}[1]{{ #1}}

\newcommand{\programmer}{Randy Mederos Santana}
\newcommand{\taskdevelop}{Desarrollo}
\newcommand{\taskdesign}{Diseño}
% Cambiar el nombre del tutor y co-tutor.
\newcommand{\tutorUNO}[1]{{\fontsize{16pt}{16pt}\selectfont #1}}
%\newcommand{\coTutor}{}

% Cambiar el cargo del tutor y co-tutor
%\newcommand{\cargoTutor}{Cargo del tutor}
%\newcommand{\cargoCoTutor}{Cargo del co-tutor}

% Cambiar la fecha de confección de la tesis. Por defecto es el día de hoy
\usepackage{datetime}


\renewcommand{\today}{\number \year} 
\newcommand{\fecha}{\large{\today}}





\newcommand{\titulo}{\tituloTesisPrimeraPagina\espacios}

\newcommand{\tipoTesis}{\nombreTipoTesis \espacios \espacios}

\newcommand{\autores}
{
	
    \begin{center}
    \begin{tabular}{rl}
    {\fontsize{16}{16} \selectfont  Autor:}  & \autorUNO{Randy Mederos Santana}  \vspace{1.5em}   \\
                       
    {\fontsize{16}{16} \selectfont Tutor:}    & \tutorUNO{M.Sc. Luis Andrés Valido Fajardo}    \\
    %\large{\bf Co-Tutor:} & \large{\coTutor}
    \end{tabular}
    \end{center}
    \vspace{1.0in}
}

\newcommand{\ciudadFecha}{{\fontsize{14pt}{14pt} \selectfont Matanzas, 2022}}

\newcommand{\fillDia}{\makebox[0.3in]{\hrulefill}\space}
\newcommand{\fillMes}{\makebox[1in]{\hrulefill}\space}
\newcommand{\fillAnno}{\makebox[0.5in]{\hrulefill}}

\newcommand{\firma}{\makebox[2in]{\hrulefill}}

\newcommand{\firmaTesis}
{
    \begin{center}
    \begin{tabular}{ccc}
             &  \firma  &     \\
             &  \autorFIRMA{Randy Mederos Santana} &   \\
    \end{tabular}
    \end{center}
}

\newcommand{\comentario}[2][]
{
	\todo[caption={#2}, size=\small, #1, inline,color={red!100!green!8},bordercolor=red,linecolor=red]
	{
	  \renewcommand{\baselinestretch}{1.2}\selectfont#2\par
	}
}

%Biliografia
%\RequirePackage[round, authoryear]{natbib}
%\usepackage[numbers]{natbib}
%\usepackage[sort,round,colon,authoryear]{natbib}
%\usepackage{lipsum}

\renewcommand{\chaptertitlename}{\uppercase{Variable}}
\renewcommand{\tablename}{Tabla}
\newcommand{\mc}[3]{\multicolumn{#1}{#2}{#3}}
% Formateo de la presentacion del titulo de cada capítulo
\titleformat{\chapter}[hang] 
{\fontsize{14pt}{14pt}\bfseries}{\chaptertitlename\ \thechapter:}{0.5em}{} 

%Quitar el margen a la primera línea de los párrafos
\setlength{\parindent}{0in}

%Formato de las secciones
\titleformat{\section}{\normalfont\fontsize{12pt}{12pt}\bfseries}{\thesection}{1em}{}
\titleformat{\subsection}{\normalfont\fontsize{12pt}{12pt}\bfseries}{\thesubsection}{1em}{}
\titleformat{\subsubsection}{\normalfont\fontsize{12pt}{12pt}\bfseries}{\thesubsubsection}{1em}{}

\setcounter{secnumdepth}{3}
\setcounter{tocdepth}{2}

\renewcommand{\tablename}{Tabla}
\def\tablename{Tabla}