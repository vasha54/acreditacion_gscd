La carrera de Licenciatura en Gestión Sociocultural para el Desarrollo, está conformada por un claustro de profesores procedentes de diversas áreas del saber, con elevado sentido de pertenencia, responsabilidad, consagración, entrega, dedicación al trabajo y compromiso con la Universidad y el proceso revolucionario, en aras de promover la mejora continua en la gestión de su calidad. 

El colectivo está integrado por un total de 46 docentes que imparten las asignaturas del plan de estudio vigente. Lo conforman graduados de la carrera y profesores con perfiles afines como Historia del Arte, Educación Plástica, Educación Musical, Español Literatura, Filosofía e Historia, Información Científico-Técnica y Bibliotecología, Comunicación Social, Cultura Física, Informática, Turismo, Sociología, entre otros. Se combina la experiencia docente con jóvenes comprometidos con la formación de las nuevas generaciones y su superación profesional. Prestan servicios a carreras de la Universidad y la Facultad como Comunicación Social y Periodismo, Lengua Inglesa, Educación Artística, Turismo y Programas de formación de ciclo corto de Trabajo social y Asistencia Turística. 
Del total de profesores del claustro pertenecen al departamento 10 y los demás, a otros departamentos de las Facultades de Ciencias Sociales y Humanidades, Cultura Física, Agronomía, Ciencias Económicas, Técnicas, Educación, y a la Dirección de Historia y Marxismo Leninismo, así como profesores a tiempo parcial que son trabajadores de las unidades docentes. 

La carrera cuenta con 2 adiestrados, vinculados a las tareas del departamento. Tiene su plan de trabajo elaborado que incluye tareas de formación profesional vinculadas a la formación pedagógica.  

El colectivo se distingue por su articulación con organismos, organizaciones, asociaciones e instituciones culturales y por su colaboración con universidades en el extranjero, además, fomenta la participación en actividades extensionistas, investigativas y el trabajo político e ideológico. 

La atención a los profesores de la Sedes Universitarias Municipales se ha estructurado en correspondencia con las necesidades de superación de los mismos, realizándose un día al mes la atención en la sede central, además de la visita a los municipios para la capacitación metodológica y el control del proceso docente, cuando se ha visto afectada por la etapa de pandemia Covid-19 o por la escasez de combustible, se realizan videoconferencias con este fin. Es importante señalar el trabajo metodológico desarrollado para la preparación del claustro a tiempo parcial y su proceso de categorización.

La carrera realiza un trabajo metodológico para desarrollar el modelo del profesional y de formación-educación en valores, así como su integración a la Estrategia Educativa, que elabora y supervisa el profesor principal de cada año (PPA), se concreta en los colectivos de disciplina, asignaturas y año y es la guía para la labor educativa, político-ideológica de estudiantes y profesores. 

El trabajo educativo en la carrera se rige por documentos y acciones metodológicas diseñadas con una marcada intención formadora. La estrategia de trabajo educativo, integra a todos los profesores del año, a los dirigentes estudiantiles de las brigadas de la FEU y comités de base de la UJC y a los militantes del PCC que atienden estos últimos. El espacio para su implementación es el colectivo de año y está soportado en la labor conjunta que desarrollan estudiantes y profesores. Este se reúne mensualmente para facilitar el seguimiento del proceso docente-educativo, su trabajo metodológico e interdisciplinario y la atención individualizada a los estudiantes, con énfasis en los que presentan mayores dificultades. 

Los profesores participan con los estudiantes, sistemáticamente, en las reuniones de brigadas y de Comité de Base (cada profesor atiende una brigada y un Comité de Base), asambleas de integralidad e incondicionalidad, proyectos comunitarios, Juegos Yumurinos, Festivales de Artistas Aficionados, Jornadas Científicas Estudiantiles, eventos nacionales e internacionales, actos políticos, matutinos, izaje de la bandera, tareas de impacto, entre otras actividades. De igual forma, los profesores participan activamente en la ejecución y/o acompañamiento de los proyectos socioculturales como: Día de la carrera y Galería de Arte Yoel Peláez y las actividades por la campaña Violencia cero.  Estos propician el trabajo con los modos de actuación del profesional. Además, se realizan diferentes acciones de extensión universitaria correspondientes a las Cátedras Honoríficas, donde se destacan la Cátedra Fernando Ortiz (CEMFO), Cátedra \emph{Nicolás Guillén}, Cátedra de Arqueología, Cátedra de Género, Cátedra Camilo Cienfuegos y la Cátedra Martiana.

El claustro se prestigia por la obtención de premios científicos y reconocimientos otorgados, sobresalen: Premio \emph{Gloria Guerra Menchero}, \emph{Academia de Ciencias de Cuba}, Medalla \emph{Distinción por la Educación Cubana}, \emph{Unión Nacional de Historiadores de Cuba}, UNEAC, CITMA, \emph{Evangelio Vivo}, Reconocimientos de la Dirección Provincial de Cultura, del Centro Provincial de Patrimonio Cultural, del  Centro de Investigaciones y Desarrollo de la Cultura \emph{Juan Marinello} y de la Asociación de Pedagogos, Premio \emph{Alma Mater}, \emph{Tiza de Oro} , \emph{Educadores Ejemplares} y \emph{Trabajadores Destacado por la Calidad}.

Lo expuesto se corrobora con el proceso de evaluación anual de los docentes en los últimos cursos, avalado por la calidad de los controles a clases, los cambios de categorías hacia categorías superiores, el quehacer científico y la formación postgraduada. Es meritorio significar el grado de satisfacción que los estudiantes manifiestan por la calidad de las actividades docentes, el nivel de preparación del claustro de la carrera y la atención a sus diferencias individuales.