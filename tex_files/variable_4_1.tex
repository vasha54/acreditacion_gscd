La infraestructura con la que cuenta la carrera Gestión Sociocultural para el Desarrollo en la Universidad de Matanzas presenta un mantenimiento de la calidad en el período evaluado. Se potencian las capacidades de los medios con que cuenta la universidad con los propios del Departamento Docente y las unidades docentes.

El 100\% de los estudiantes cuenta con bibliografía actualizada de manera constante por los profesores en las diferentes asignaturas del currículo base, propio y optativo / electivo, lo que contribuye a su correcta formación como futuro profesional. Es de destacar que esta bibliografía se encuentra tanto en formato papel en la biblioteca de la universidad, como en formato digital en diferentes espacios.

También se benefician de los servicios que ofrecen los fondos de los centros de documentación provinciales: Biblioteca \emph{Gener y Del Monte}, Centro de Superación para la Cultura \emph{Ignacio Cervantes}, Museo Provincial Palacio de Junco, Consejo Provincial de las Artes Escénicas. Además de acceder a los servicios que brindan las bibliotecas públicas de sus municipios.

El vínculo de profesores del claustro con instituciones extranjeras, como la Universidad de Valencia (España), permite el acceso a sus bibliotecas virtuales, posibilitando la descarga de bibliografía actualizada tanto para el proceso docente como la superación profesional del claustro. También se mantiene un constante intercambio tanto en el pregrado como en el posgrado.

Esta bibliografía de la carrera se complementa con materiales audiovisuales (películas, documentales, cortos) que son material de apoyo a la docencia y están disponibles en el \href{http://rein.umcc.cu/}{Repositorio de la UM}( donde se ubica el repositorio de tesis, en el cual se archivan las tesis y tesinas realizadas en los ejercicios de culminación de estudios, de pregrado y posgrado. En este repositorio los estudiantes pueden buscar antecedentes que aborden temas relacionados con sus estudios) , el Cine Club, y algunos en los \href{https://eva.umcc.cu/}{Entornos Virtuales Aprendizaje (EVA)} de la universidad (según capacidad del material). Estos materiales suman más de 500 audiovisuales disponibles para el uso de los estudiantes y profesores. También, junto a la creación de usuarios para el acceso a los servicios en plataformas digitales, se crea una cuenta de correo electrónico institucional, que cada profesor y estudiante puede utilizar para la comunicación.

Se promueve de igual manera la consulta de artículos científicos en sitios de uso a nivel internacional como:

\begin{itemize}
	\setlength\itemsep{-0.5em}
	\item \href{https://www.researchgate.net}{ResearchGate}
	\item \href{https://www.scielo.org/es}{SciELO.org}
	\item \href{https://scholar.google.com/}{Google Académico}
	\item \href{https://dialnet.unirioja.es}{Dialnet}
	\item \href{https://www.academia.edu}{Academia.edu}
\end{itemize}

En el sitio institucional de la universidad, al cual se accede mediante el siguiente \href{https://www.umcc.cu/}{link}, se puede encontrar el \href{http://cict.umcc.cu/}{Centro de Información Científico Técnica (CICT)} , plataforma que permite acceder a los diferentes motores de búsquedas de información existentes en Internet.

Se cuenta con un sistema de descarga de softwares interno necesarios para la informatización del proceso docente. Desde dicha plataforma se pueden obtener aplicaciones y programas, tanto de seguridad con antivirus como editores de texto, sistemas operativos, entre otros. A continuación, se mencionan algunos de los servicios que ofrece:

\begin{itemize}
	\setlength\itemsep{-0.5em}
	\item Herramientas digitales de servidores o documentos (MobaXterm, EndNote)
	\item Sistema Operativo (Linux, Mac, Windows)
	\item Traductores (IdiomaX Translation Suite, Babylon, Power Translator)
	\item Programas para desarrollo de aplicaciones informáticas ( Docker, GIT, GITLAB, NET Core)
	\item Documentales
	\item Gabinetes (Antivirus, Math Lab, PSIM, QGIS) 
\end{itemize}

El 100\% de las asignaturas de la carrera se encuentran montadas en los \href{https://eva.umcc.cu/}{Entornos Virtuales de Aprendizajes} de la universidad donde se ofrecen materiales, guías de estudio, recursos educativos, que apoyan el proceso docente y donde los estudiantes de la carrera acceden en su totalidad. Mediante estos servicios accesibles desde cualquier dispositivo con conectividad, se complementan las actividades presenciales y se agilizan las orientaciones del proceso de enseñanza-aprendizaje.