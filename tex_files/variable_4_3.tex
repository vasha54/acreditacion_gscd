Dentro de la Facultad de Ciencias Sociales y Humanidades de la Universidad de Matanzas se cuenta con un aula especializada, destinada al trabajo con los estudiantes en conferencias especializadas, realización de talleres científicos, exposición de trabajos, trabajo de las cátedras honoríficas y multidisciplinarias, entre otros. El Edificio A, donde se encuentra ubicada la facultad, tiene una sala de navegación con máquinas y cables de red para garantizar la conectividad.

Esto se complementa con los laboratorios docentes, en los cuales se pueden realizar clases, consultas de bibliografía, talleres de capacitación, donde pueden acceder estudiantes y profesores. Los laboratorios son un total de 7, que cuentan con una cantidad de 59 computadoras, además de puestos disponibles para el uso de cables de red en el caso del personal que cuenta con dispositivos propios. La distribución de los laboratorios es de 6 para uso común y 1 para profesores y la realización de posgrados, y prestan servicios las 24 horas del día. 

\begin{longtable}{|c|c|p{8cm}|p{4cm}|}
\hline
\underline{\textbf{Local}}	&  \underline{\textbf{Cant. de PC}} & \multicolumn{1}{c}{\underline{\textbf{Actividades}}} & \multicolumn{1}{|c|}{\underline{\textbf{Observaciones}}} \\ \hline
Laboratorio 3	& 13 & Docencia y tiempo de máquina para & \\ \cline{1-2} \cline{4-4}
Laboratorio 4	& 13 &  estudiantes y profesores & \\ \cline{1-2} \cline{4-4}
Laboratorio 5	& 10 & & \\ \cline{1-2} \cline{4-4}
Laboratorio 6	& 8 & & \\ \hline
Laboratorio 7	& 15 & Docencia de postgrado y tiempo de máquina para profesores & Con climatización \\ \hline
	\caption{Laboratorios de la Universidad de Matanzas (Elaboración propia)}
\end{longtable}

La cuota de uso para la navegación de estudiantes y personal docente ha aumentado en los últimos años. La capacidad de navegación ahora presenta una asignación de entre 2 y 7 GB para los estudiantes, aumentado desde que ingresan y a medida que avanzan en la formación; los profesores tienen desde 9 GB los Auxiliares Técnicos Docentes, hasta 19 GB los profesores con categoría docente principal.

\begin{longtable}{|c|c|c|c|}
	
		\endfirsthead
	
	\mc{4}{>{}c}{\tablename\ \thetable{} Continuación de la página anterior }\\ 
	
	\endhead
	\hline
	\underline{\textbf{Año}} & \underline{\textbf{Matrícula}} & \underline{\textbf{Cant. Cuentas}} & \underline{\textbf{Cant. Automáticas}} \\ \hline
	Primero & 22 & 22& 17 \\ \hline
	Segundo & 13 & 12 &2 \\ \hline
	Tercero & 12 & 12&0 \\ \hline
	Cuarto & 15 & 15  & 2 \\ \hline
	\textbf{Total} & 62 & 61 & 21 \\ \hline
	\caption{Cuentas de usuarios habilitada para estudiantes}
\end{longtable}

Existen en la universidad alternativas para el acceso a internet y otras plataformas, mediante el uso de zonas WiFi, a partir de un registro previo de los dispositivos que se van a utilizar. Estas zonas están repartidas por el espacio universitario y se encuentran en el laboratorio de profesores, el pasillo del laboratorio, en el Parque de la Juventud, en la biblioteca, Rectoría y el motel de la universidad. 

El personal cuenta con vías alternativas para acceder de forma externa a algunos servicios que brindan nuestras plataformas, entre los que se destacan el correo institucional, los \href{https://eva.umcc.cu/}{Entornos Virtuales Aprendizaje (EVA)}, el \href{http://cict.umcc.cu/}{CICT} y el Catálogo en línea de la biblioteca. Estos servicios, debido a convenios del MES con ETECSA, son exonerados de pagos, no necesitan autentificar usuario nauta ni paquetes de datos activos. El personal dispone de sistemas de préstamo de bibliografía, tanto en la biblioteca de la Universidad de Matanzas, como en el almacén de textos.

La actividad docente se realiza en aulas asignadas en los edificios docentes y aulas especializadas, con un buen estado constructivo, de mantenimiento reciente y baños restaurados. Las aulas cuentan con pizarras y tizas, además los profesores disponen de cuatro televisores en el Departamento Docente y Datashow. Los profesores y estudiantes poseen celulares y computadoras que utilizan para realizar actividades que facilitan el proceso docente, complementando la infraestructura propia de la carrera. La relación de los medios por estudiantes se muestra en la siguiente tabla:

\begin{longtable}{|c|c|c|c|c|}
	\hline
	 \underline{\textbf{Año}} & \underline{\textbf{Matrícula}} & \underline{\textbf{Laptop (\%)}} & \underline{\textbf{Celulares inteligentes (\%)}} & \underline{\textbf{PC de mesa (\%)}}\\ \hline
	 Primero & 12 & 8 (66,7\%)& 11 (91,67\%) & 5 (41,67\%)\\ \hline
	  Segundo & 11 & 5 (45,45\%) &11 (100\%) & 2 (18,18\%)\\ \hline
	   Tercero & 12 & 9 (75\%)&12 (100\%) & 3 (25\%)\\ \hline
	    Cuarto & 8 & 7 (87,5\%) & 8 (100\%) & 2 (25\%)\\ 
	\hline
\end{longtable}