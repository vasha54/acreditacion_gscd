Los estudiantes de la carrera reconocen que un papel fundamental en el proceso de formación lo ejercen los profesores del claustro. Las encuestas de satisfacción del profesional aplicadas por la dirección de formación del profesional del MES muestran en este aspecto resultados satisfactorios. La carrera posee un diseño para el sistemático seguimiento a los egresados, con visitas a los centros de trabajo, encuentros con egresados, la aplicación de encuestas para conocer las debilidades en su formación, que permitan el perfeccionamiento y actualización del plan de estudio en las diferentes cohortes, así como el diseño de la superación postgraduada que permita minimizar sus deficiencias.  

Por medio de entrevistas a empleadores se reconoce que los egresados de la carrera Gestión Sociocultural para el Desarrollo, en las diferentes instituciones y la sociedad en general, manifiestan las siguientes características: 

\begin{itemize}
	\setlength\itemsep{-0.5em}
	\item Dominan las habilidades investigativas y de dirección en su mayoría, con énfasis en la teoría estudiada en sus años de formación.
	\item Identifican los potenciales culturales del entorno social en que se desenvuelven y actúan en correspondencia con ello.
	\item Realizan diagnósticos situacionales de problemáticas, fortalezas artísticas, sondeos de gustos y preferencias de público, compilación de bienes e inmuebles con valor arquitectónico para las localidades, entre otros
	\item Participan en equipos multidisciplinarios para llevar a cabo investigaciones, enfatizando en problemáticas de desarrollo y transformación social
	\item Realizan acciones de gestión sociocultural en su sentido más amplio, lo que demuestra la formación adquirida durante la carrera
	\item Realizan labores docentes dentro del Movimiento de Alumnos Ayudantes y el Contingente Educando por Amor
\end{itemize}

En este último aspecto la facultad y la universidad en sentido general, manifiestan satisfacción con los egresados de la carrera, estos se han incorporado a la docencia universitaria (10), algunos ya han alcanzado, la categoría de Asistentes y Auxiliares, poseen categoría científica y el 100\% está vinculado a una modalidad de postgrado, son másteres, o se encuentran en programas de maestría o doctorado.

En las instituciones provinciales y municipales del MINCULT, sector principal donde se ubican los egresados de la carrera, los empleadores manifiestan satisfacción con la calidad de los egresados.\\

\textbf{Fortalezas:}

\begin{itemize}
	\setlength\itemsep{-0.5em}
	\item Relevante proyección de la carrera hacia el territorio a partir de la participación de los profesores y estudiantes en la solución de los problemas mediante la actividad investigativa coordinada a través de las líneas de investigación de la carrera, facultad y universidad. Igualmente es elevada la participación en tareas de impacto social donde ponen en práctica sus modos y esferas de actuación profesional como: participación en estudios territoriales y nacionales sobre adolescencias y juventudes, Movimiento Sembrar ConCiencia, labores de apoyo a la erradicación de la Covid-19 y acciones de trabajo comunitario integrado.
	\item Elevada satisfacción de los egresados con su proceso de formación profesional, lo que se demuestra a partir del seguimiento al egresado que se realiza desde la carrera, con entrevistas a directivos de los organismos empleadores. Ellos manifiestan que son profesionales coherentes y con una elevada cultura de la profesión.
	\item Elevada presencia de profesores y estudiantes en proyectos de investigación que dan respuestas a demandas del territorio como: Gestión sociocultural para el desarrollo local en el Consejo Popular Matanzas Este; VIDAS en el Rabí; Proyecto identidad y realidad cubana: estudio sociocultural del impacto de las transformaciones socioeconómicas en el centro de Cuba (UCF).
	\item Destacada presencia de profesores y estudiantes en redes nacionales e internacionales, como REDIPE, Asociación de Pedagogos de Cuba y la Unión de Historiadores de Cuba.
\end{itemize}

\textbf{Debilidades:}
\begin{itemize}
	\setlength\itemsep{-0.5em}
	\item No se declaran
\end{itemize}