Los estudiantes de la carrera reconocen que un papel fundamental en el proceso de formación lo ejercen los profesores del claustro. Las encuestas de satisfacción del profesional aplicadas por la dirección de formación del profesional del MES muestran en este aspecto resultados satisfactorios. La carrera posee un diseño para el sistemático seguimiento a los egresados, con visitas a los centros de trabajo, encuentros con egresados, la aplicación de encuestas para conocer las debilidades en su formación, que permitan el perfeccionamiento y actualización del plan de estudio en las diferentes cohortes, así como el diseño de la superación postgraduada que permita minimizar sus deficiencias. 

Por medio de entrevistas a empleadores se reconoce que los egresados de la carrera Gestión Sociocultural para el Desarrollo, en las diferentes instituciones y la sociedad en general, manifiestan las siguientes características: 

\begin{itemize}
	\item Dominan las habilidades investigativas y de dirección en su mayoría, con énfasis en la teoría estudiada en sus años de formación.
	\item Identifican los potenciales culturales del entorno social en que se desenvuelven y actúan en correspondencia con ello.
	\item Realizan diagnósticos situacionales de problemáticas, fortalezas artísticas, sondeos de gustos y preferencias de público, compilación de bienes e inmuebles con valor arquitectónico para las localidades, entre otros.
	\item Participan en equipos multidisciplinarios para llevar a cabo investigaciones, enfatizando en problemáticas de desarrollo y transformación social.
	\item Realizan acciones de gestión sociocultural en su sentido más amplio, lo que demuestra la formación adquirida durante la carrera.
	\item Realizan labores docentes dentro del Movimiento de Alumnos Ayudantes y el Contingente Educando por Amor.
\end{itemize}

En este último aspecto la facultad y la universidad en sentido general, manifiesta satisfacción con los egresados de la carrera, estos se han incorporado a la docencia universitaria (10), algunos ya han alcanzado, la categoría de Asistentes y Auxiliares, poseen categoría científica y el 100\% está vinculado a una modalidad de postgrado, son másteres, o se encuentran en programas de maestría o doctorado.

En las instituciones provinciales y municipales del MINCULT, sector principal donde se ubican los egresados de la carrera, los empleadores manifiestan satisfacción con la calidad de los egresados.\\

\textbf{Fortalezas:}

\begin{itemize}
	\item Destacada participación de los profesores y estudiantes en la solución de los problemas del territorio, mediante la actividad investigativa coordinada a través de las líneas de investigación de la carrera, facultad y universidad. Igualmente es elevada la participación en tareas de impacto social donde ponen en práctica sus modos y esferas de actuación profesional.
	\item Elevada satisfacción de los egresados con su proceso de formación profesional, lo que se demuestra a partir del seguimiento al egresado que se realiza desde la carrera, con entrevistas a directivos de los organismos empleadores. Ellos manifiestan que son profesionales coherentes y con una elevada cultura de la profesión.
	\item Los estudiantes participan en la solución de los problemas del territorio, vinculados a su profesión. Destacándose la participación en INNOVAMATANZAS, 2018, aplicando encuestas a la población y presentándose los resultados de esta tarea en el taller de cierre del evento. En el curso 2019-2020, se encuestaron a jóvenes matanceros para un estudio nacional sobre adolescencias y juventudes, coordinado por la UJC. En el curso 2022, los estudiantes de 4to año participaron en las actividades realizadas en la provincia de Matanzas por el Movimiento Sembrar Con Ciencia. Se destacaron en las labores en apoyo a la COVID-19, incorporados a las pesquisas, a la zona roja, al trabajo en organopónico y a otras actividades convocadas para el enfrentamiento a la pandemia. También destacan las acciones de apoyo comunitario integrado realizadas durante el curso 2021.
	\item El 100\% del trabajo científico-investigativo y profesional que realizan los estudiantes responde a las problemáticas principales de las instituciones del territorio y fortalece la pertinencia social de la carrera, relacionándose con los modos y esferas de actuación del profesional. Demostrado en que el 100\% de los estudiantes en el curso 2022 se encuentran incorporados a proyectos de investigación donde desarrollan investigaciones sobre temas desarrollo local e intervención comunitaria, procesos culturales cubanos, extensión universitaria, gestión integral del patrimonio; con resultados científicos que se exponen en eventos, jornadas científicas y se presentan como tema de trabajos de diploma. El 100\% de los estudiantes forman parte de las sociedades científicas estudiantiles.
\end{itemize}

\textbf{Debilidades:}
\begin{itemize}
	\item No se declaran
\end{itemize}