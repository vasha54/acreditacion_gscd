El Departamento de Estudios Socioculturales, donde radica la carrera Gestión Sociocultural para el Desarrollo, durante los últimos cinco años ha logrado afianzar los vínculos de los profesores y estudiantes con la solución de los problemas del territorio y/o del país. De esta manera se cumple con el principal encargo de la carrera, que es formar profesionales centrados en los procesos de gestión sociocultural dirigidos a potenciar el desarrollo humano, individual y colectivo, que incide directamente en el enriquecimiento espiritual, fortalecimiento de la identidad cultural y nacional, en la calidad de la vida colectiva y la capacidad de participación de la población en el desarrollo social, en correspondencia con los Lineamientos de la Política Económica y Social del Partido y la Revolución, el Plan Nacional de Desarrollo Económico y Social 2030 y los 17 objetivos para el desarrollo sostenible declarados en la Agenda 2030. 

En la provincia de Matanzas la carrera está abierta en la Universidad de Matanzas (sede central) y en siete de sus municipios: Cárdenas, Colón, Jovellanos, Jagüey Grande, Calimete, Limonar y Los Arabos. La labor que realiza desde cada una de sus sedes, hace que posea una elevada pertinencia, avalada por la contribución al estudio, tratamiento y solución a problemas del territorio, con la participación de los estudiantes y profesores del claustro.

El trabajo con el banco de problemas del territorio se organiza desde el colectivo de carrera, colectivo de la Disciplina Principal Integradora y los colectivos de año, con el protagonismo de la FEU en las Sociedades Científicas Estudiantiles (SCE), en correspondencia con el objetivo integrador de cada año académico y con las líneas de investigación. A partir de estas, teniendo en cuenta el trabajo interdisciplinar, los modos y esferas de actuación profesionales, las exigencias del modelo del profesional y la ejecución del componente laboral investigativo se realizan las orientaciones de la actividad científica de profesores y estudiantes.

Constituye un elemento fundamental en la carrera la formación de cuatro SCE en líneas de investigación, que trabajan las siguientes temáticas: desarrollo local e intervención comunitaria, procesos culturales cubanos, extensión universitaria y gestión integral del patrimonio. A estas temáticas se asocian los estudiantes de la carrera quienes unidos a sus profesores/tutores desarrollan acciones desde la docencia, la investigación y la extensión.

A partir de lo expuesto, se realizan trabajos de curso e investigaciones que aportan resultados científicos que tributan a proyectos de investigación y generalmente se concretan en trabajos de diploma. Materializadas en propuestas de soluciones totales o parciales a problemáticas relacionadas a la gestión sociocultural, donde los estudiantes ponen en práctica sus modos y esferas de actuación profesionales para la prevención de salud, medioambiental, de género, social, la gestión turística, la gestión cultural de instituciones, gestión sociocultural del patrimonio, de la información y el conocimiento, entre otras.

Hasta la fecha se han presentado 3 convocatorias a ejercicios de culminación de estudios del Plan de Estudio E, en la carrera Gestión Sociocultural para el Desarrollo, para un total de 36 graduados. De ellos, en el curso 2019-2020 y curso 2021, fueron eximidos estudiantes por resolución rectoral en período de Covid 19 y hubo presentados en examen estatal, por lo que, de un total de 27 egresados en estos cursos, 13 (48,1\% con respecto a la cantidad de egresados), presentaron Trabajos de Diploma que aportan resultados científicos en sistemas de actividades y planes de acciones para implementarse en instituciones educativas, instituciones culturales, comunidades, medios de comunicación (la radio), en el sector no estatal y en el sector del turismo, en los siguientes temas: 

\begin{itemize}
	\setlength\itemsep{-0.5em}
	\item El proceso de gentrificación en el poblado de Boca de Camarioca del municipio Cárdenas.
	\item Gestión medioambiental en los niños con discapacidad intelectual del 2do ciclo de la Institución Educativa \emph{Franklin Gómez}.
	\item Gestión turística del patrimonio cultural y participación comunitaria en el Consejo Popular René Fraga Moreno de Colón.
	\item La participación de los niños con discapacidad intelectual de segundo ciclo de la Institución Educativa Franklin Gómez que contribuya a la integración con la comunidad.
	\item Gestión turística en el proceso de rehabilitación de la Zona Priorizada para la Conservación del Centro Histórico Urbano de la ciudad de Matanzas.
	\item Inserción social de pacientes rehabilitados. Estudio de caso: Área de Neurodesarrollo.
	\item La emisora Radio Victoria de Girón como vía de formación cultural para los adolescentes jagüeyenses
	\item Política Cultural Cubana en el sector no estatal: caso de estudio CNA Decorarte
	\item La promoción del Taller de Creación Literaria Pablo Neruda de la Universidad de Matanzas (2018-2021)   
	\item Prevención de la violencia intrafamiliar en la Comunidad de Santa Ana, municipio Limonar, provincia Matanzas 
	\item La gestión interpretativa del patrimonio urbano matancero para el desarrollo del turismo cultural.
	\item Los estudios de violencia de género desde la extensión universitaria en la Universidad de Matanzas.   
	\item Propuesta de recorrido interpretativo en el Museo Juan Gualberto Gómez de Unión de Reyes.
\end{itemize}

En el curso 2022, de un total de 9 estudiantes evaluados, (66, 7\% con respecto a la cantidad de egresados), presentan trabajos de diploma, con los temas siguientes:

\begin{itemize}
	\setlength\itemsep{-0.5em}
	\item Turismo literario en las editoriales matanceras mediante la interpretación del patrimonio 
	\item Las TIC en la socialización del patrimonio en el Centro Histórico Urbano de Matanzas
	\item Los Círculos de interés: una vía para la educación patrimonial en las escuelas primarias
	\item La educación patrimonial para el desarrollo sociocultural desde el museo \emph{Constantino Barredo Guerra} de Perico
	\item Gestión de la comunicación en las actividades de extensión universitaria
	\item Participación comunitaria de las personas que viven con discapacidad físico motoras en las actividades culturales
\end{itemize}

Como se ha planteado, la participación y/o coordinación de proyectos de investigación por parte de profesores del claustro posibilita la inserción de estudiantes e investigadores, asume la fortaleza multidisciplinaria de los mismos y por ello contribuye y beneficia el trabajo interdisciplinar. En dichos proyectos se han ejecutado investigaciones que tributan a resultados científicos presentados en eventos internacionales (UNISOC - Universidad Sociedad), en talleres, Jornadas Científicas Estudiantiles, en ejercicios de culminaciones de estudio, etc. Proponen también la formación posdoctoral, doctorales y de maestría. Entre los proyectos que se destacan por sus resultados se encuentran:

\begin{itemize}
	\setlength\itemsep{-0.5em}
	\item Proyecto Estudios socioculturales para el desarrollo sostenible: UM-DECORARTE (PNAP). Dr. C. Rosa Elvira Alfonso Ramos. (se propone la evaluación del impacto sociocultural de la Cooperativa No Agropecuaria DECORARTE, en la comunidad intra y extra universitaria)
	\item Gestión sociocultural para el desarrollo local en el Consejo Popular Matanzas Este (PAPT). Dr. C. Ana Gloria Peñate Villasante. (atiende un problema de investigación relacionado con el apoyo al diseño y manejo de destinos turísticos sostenibles. Se plantea como objetivo general contribuir al desarrollo del turismo cultural a partir de la gestión turística del patrimonio en el Consejo Popular Matanzas Este, específicamente en la zona declarada Monumento Nacional (MN))
	\item VIDAS en el Rabí (PAPT). Dr. C. Odalis Alberto Santana. (responde a una problemática demandada por AZCUBA, ante la necesidad de reanimar el desarrollo sociocultural en la comunidad Jesús Rabí, donde se encuentra uno de los polos productivos más importantes del país, corre el riesgo de la pérdida de fuerza laboral, entre otras causas, por la emigración de sus habitantes hacia otros territorios. Tiene como objetivo general contribuir al desarrollo sociocultural de la comunidad Jesús Rabí en el municipio de Calimete; a partir de la articulación del trabajo comunitario y la gestión sociocultural en este sitio)
	
\end{itemize}

Se cuenta con otros proyectos de investigación, donde participan profesores y estudiantes del claustro, como por ejemplo:

\begin{itemize}
	\setlength\itemsep{-0.5em}
	\item Proyecto Identidad y realidad cubana: estudio sociocultural del impacto de las transformaciones socioeconómicas en el centro de Cuba (PAP) (I+D+I). M. Sc. Soilen Cedeño Solís. (Con el propósito de fundamentar desde la perspectiva sociocultural el impacto de las transformaciones socioeconómicas en la identidad cultural en el centro de Cuba)
	\item La Competencia Comunicativa Intercultural en el Discurso de Interpretación Patrimonial para el Desarrollo Local del Turismo de Ciudad. (PNAP, 2022-2024). Dr. C. Jorge Luis Rodríguez Morell. (Con el objetivo de contribuir a la construcción progresiva de la competencia comunicativa intercultural en el discurso de interpretación patrimonial de los actores comunicativos fundamentales asociados al desarrollo local del turismo de ciudad)
	\item Patrimonio cultural y formación: patrimonio cultural universitario (PCU), historia, educación patrimonial y desarrollo local (PNAP, 2022-2024). Dr. C. Lissette Jiménez Sánchez. (Se propone como resultado principal un programa educativo, que valoriza el PCU y significa su contribución a la formación integral y diversificada de profesionales en la Universidad de Matanzas)
	Perfeccionamiento de la Gestión Universitaria (PNAP). Dr. C. Lourdes Tarifa Lozano. (Con el objetivo de perfeccionar el sistema de gestión universitaria orientado a la calidad que facilite el funcionamiento adecuado de la institución de educación superior, teniendo en cuenta tanto el marco legal como la responsabilidad social establecida y vinculado al trabajo inherente del sistema de dirección)
\end{itemize}

En el caso de los proyectos socioculturales A Moverse por el Cambio, en el de la Galería Joel Peláez y AfroAtenas, aportaron resultados para eventos, Jornadas Científicas Estudiantiles y trabajos de diploma en los cursos 2017-2018, 2018-2019 y 2019-2020.

Se aprecia un gradual incremento en la superación y formación de los docentes del Departamento de Estudios Socioculturales. Como resultado de este proceso continuo, 3 profesoras se encuentran cursando maestrías que tributan a: Administración de Empresas, Didáctica de las Humanidades y en Estudios Sociales y Comunitarios; en formación doctoral se encuentran 5 profesores: 3 en Ciencias de la Educación, 1 en Ciencias Históricas y 1 en Ciencias Económicas. Esta superación continua incrementa el impacto de la carrera hacia el territorio matancero

Existe una estrecha colaboración con las entidades del territorio, entre ellas: la Dirección Provincial de Cultura, el Centro de Superación para la Cultura, la Delegación Territorial de Ciencia Tecnología y Medio Ambiente (CITMA), Museo de Arte Matanzas, Ediciones Vigía, Asociación Hermanos Saiz, Dirección Provincial de Salud, Dirección Provincial de Trabajo y Seguridad Social, la Federación de Mujeres Cubanas, Centro Provincial de Patrimonio Cultural, Oficina de Monumentos y Sitios Históricos, Oficina del Conservador, Teatro Sauto, Archivo Histórico Provincial de Matanzas, entre otras.

Estas relaciones institucionales han posibilitado se evidencie el impacto social de la carrera en la provincia, ya que ha diversificado la mirada en torno a la formación que recibe el egresado de la carrera Gestión Sociocultural para el Desarrollo.

El Taller Cultura-Universidad es un ejemplo de los vínculos entre organismos empleadores, la carrera y los egresados de la misma. Este evento, que cuenta con más de quince ediciones, se ha convertido en un espacio de integración entre la primera unidad docente de la carrera, el Grupo de Investigación y Desarrollo de la Dirección Provincial de Cultura (Unidad de Desarrollo e Innovación) y otros Organismos de la Administración Central del Estado (OACE).

Se desarrollan actividades en proyectos de investigación, maestrías, diplomados, especialidades y doctorados vinculados a diferentes instituciones sociales y educacionales, tales como: el Departamento de Estudios de Ciencias de la Educación de la Universidad de Matanzas (DEDES), Facultad de Ciencias de la Educación, Facultad de Ciencias Técnicas, el Centro de Estudios de Anticorrosivos y Tensoactivos, Centro de Información Científico Técnico y Departamento de Historia y Marxismo - Leninismo, entre otras.

De igual modo se fortalece el impacto social en las instituciones y la formación posgraduada, con los posgrados impartidos por el Departamento de Estudios Socioculturales en el período evaluado:\\

\underline{\textbf{Año 2018:}}

\begin{itemize}
	\setlength\itemsep{-0.5em}
	\item Posgrado: Gestión y Desarrollo cultural en el ámbito de los museos. Profesora: Dr. C. Ana Gloria Peñate Villasante. 
	\item Posgrado: Gestión de marketing en la comercialización de la cultura. Profesor: M. Sc. Andrés Rodríguez Reyes.
\end{itemize}

\underline{\textbf{Año 2019:}}

\begin{itemize}
	\setlength\itemsep{-0.5em}
	\item Posgrado: ¿Cómo escribir un documento científico? Profesora: Dr. C. Rosa Elvira Alfonso Ramos.
	\item Posgrado: Las religiones populares cubanas de origen africanas en Matanzas. M. Sc. Profesor: Andrés Rodríguez Reyes. 
\end{itemize}

\underline{\textbf{Año 2020:}}

\begin{itemize}
	\setlength\itemsep{-0.5em}
	\item Posgrado: Principios Fundamentales de la museología. Profesora: Dr. C. Ana Gloria Peñate Villasante.
	\item Posgrado: Las religiones populares cubanas de origen africanas en Matanzas. Profesor: M. Sc. Andrés Rodríguez Reyes.
	\item Posgrado: Gestión de marketing en la comercialización de la cultura. Profesor: M. Sc. Andrés Rodríguez Reyes. 
\end{itemize}

\underline{\textbf{Año 2021:}}

\begin{itemize}
	\setlength\itemsep{-0.5em}
	\item Posgrado: Gestión integral del Patrimonio cultural. Profesora: Dr. C. Ana Gloria Peñate Villasante.
	\item Posgrado: Gestión de marketing en la comercialización de la cultura. Profesor: M. Sc. Andrés Rodríguez Reyes.
\end{itemize}

\underline{\textbf{Año 2022:}}

\begin{itemize}
	\setlength\itemsep{-0.5em}
	\item Posgrado: Teoría y Práctica de la Interpretación del Patrimonio Profesores: Dr. C. Ana Gloria Peñate Villasante y Lic. Guillermo Alfredo Jiménez Pérez.
	\item Posgrado: Diseño Teórico - Metodológico de la Investigación Social Profesora: Dr. C. Ana Gloria Peñate Villasante.
	\item I Edición del Diplomado en Gestión Sociocultural para el Desarrollo Local y Humano. Claustro de Profesores del Departamento de Estudios Socioculturales.
\end{itemize}

Se comenzó en el curso 2022 el Programa de Técnico Superior de Ciclo Corto en Trabajo Social, siendo su matrícula los trabajadores sociales del municipio de Matanzas, en estrecho vínculo y dando respuesta a las necesidades de la Dirección Provincial de Trabajo y Seguridad Social en el territorio. 
 
La carrera cuenta en su claustro con profesionales expertos en organismos nacionales, en tribunales de cambio de categoría docente y mínimo de doctorado, miembros del Consejo Científico de la Universidad y miembros de la Junta de Acreditación Nacional (JAN), Coordinador Nacional del programa de posgrado en la Red de Educación Superior, miembro del comité nacional de expertos de Tecnología Educativa del MES, miembros de la COPED y miembros de la Comisión de Grado Científico de la UM.
 
Los egresados de la carrera que imparten docencia pueden optar por la Maestría Didáctica de las Humanidades y la Maestría en Estudios Sociales y Comunitarios, en la que parte de su Comité Académico y profesores son docentes del claustro de la carrera. También se mantiene la convocatoria del Diplomado en Gestión Sociocultural para el Desarrollo Local y Humano.

Los profesores y estudiantes han participado en tareas de impacto, entre ellas: 
acciones recuperativas de eventos meteorológicos, en la gestión de las muestras y exposiciones que se presentan en la Galería Joel Peláez, ubicada en el Departamento de Estudios Socioculturales. Han contribuido a diagnósticos con la aplicación de encuestas a la población, como fue el caso de INNOVA-MATANZAS en el curso 2018, en el curso 2019-2020 se encuestaron a jóvenes matanceros para un estudio nacional sobre adolescencias y juventudes, coordinado por la UJC. Ya en el curso 2022 se promueve a nivel nacional el Movimiento Sembrar ConCiencia, donde participaron los estudiantes de 4to año en las actividades realizadas en la provincia de Matanzas. 
En el periodo de la COVID-19 los profesores y estudiantes estuvieron incorporados a las pesquisas, a la zona roja, al trabajo en organopónico, atención a personas vulnerables, trabajo social con discapacitados y a otras actividades convocadas para el enfrentamiento a la pandemia.

El trabajo comunitario es otro de los aspectos de interés en las tareas de impacto social, por lo que profesores y estudiantes del claustro han estado presentes en el apoyo a las comunidades en programas del municipio de Matanzas de transformación social. Específicamente en la comunidad 43 del Consejo Popular Playa y en el Consejo Popular Peñas Altas. 

Se ha continuado el trabajo con el programa de universalización de la enseñanza en la provincia, tanto en las actividades metodológicas, capacitaciones, entrenamiento para la preparación científica y metodológica de los profesores en las asignaturas correspondientes.

Profesores del claustro pertenecen al Proyecto Antenas que comienza a finales del 2022, Proyecto del Observatorio Social y Laboral en las Universidades, para estudios sociales que rectorea el Ministerio de Trabajo y Seguridad Social, tributando a contrarrestar problemáticas sociales y laborales del territorio.

Se destaca de igual modo, la participación de los estudiantes en el contingente Educando pon Amor. En el curso 2017-2018 participaron 10 estudiantes, en el curso 2018-2019, 7 y en el curso 2022, 3 estudiantes.

Los aspectos anteriormente expuestos demuestran la activa participación de la carrera Gestión Sociocultural para el Desarrollo de la Universidad de Matanzas en las tareas de impacto social vinculadas al Modelo del Profesional de la carrera, resultado del cohesionado trabajo educativo y científico del claustro, con el objetivo de dar respuesta a las problemáticas del país y del territorio.