La producción científica se ha mantenido estable y se emplea como literatura básica y/o complementaria, o como materiales de consulta. De igual forma, se ha incrementado la participación en eventos por parte de los profesores del claustro. El 100\% de los profesores tienen publicaciones referenciadas y participación en eventos, ya sean territoriales, nacionales e internacionales. Es necesario destacar que el período evaluado comprende la etapa donde se estuvo inmerso en la lucha contra la Covid-19, lo que provocó una disminución en la realización de eventos científicos. Entre los eventos más significativos en los que ha participado el claustro de la carrera se pueden mencionar los siguientes: Talleres Internacionales de La enseñanza de las disciplinas Humanísticas, Convención Científica Internacional Universidad Integrada e Innovadora (CIUM), CIT@tenas 2019, Congreso Internacional de Educación y Pedagogía (REDIPE), Evento Internacional Universidad Sociedad (UniSoc), Eventos de la Asociación de Pedagogos de Cuba, Taller Nacional Museología y Sociedad, Encuentro Internacional de Investigadores de Ciencias Humanísticas, Encuentro Provincial de Patrimonio Histórico Azucarero, Simposio Internacional Desafío en el manejo y gestión de ciudades, Convención Científica Internacional de la Universidad de Cienfuegos.