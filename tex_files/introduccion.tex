%\chapter*{\large Introducci\'on}
%\addcontentsline{toc}{chapter}{\large Introducción}
%
\chapter*{Introducción}
\addcontentsline{toc}{chapter}{Introducción}



 

 
% \addcontentsline{toc}{part}{Introducci\'on}


%Sirve de presentación del trabajo y orienta al lector sobre los aspectos fundamentales de este. Se explica con absoluta claridad, de forma corrida y en no más de 5 páginas, los siguientes aspectos:

%\begin{itemize}
% \item Problema a resolver.
% \item Actualidad y necesidad del trabajo.
% \item Antecedentes (deben indicarse las referencias bibliográficas utilizadas como fuente de información).
% \item Aportes prácticos esperados del trabajo.
% \item Objeto de estudio.
% \item Campo de acción.
% \item Objetivos del trabajo (objetivo general y objetivos específicos)
% \item Tareas desarrolladas para cumplir los objetivos.
% \item Estructuración del contenido con una breve explicación de sus partes.
%\end{itemize}

En el panorama actual de la actividad turística, las Tecnologías de la Información y las Comunicaciones (TIC´s) han logrado un rol protagónico en las proyecciones del sector. La Organización Mundial del Turismo, en su Programa de Asistencia Técnica para la Recuperación del Turismo de la Crisis de la Covid19, propone en uno de sus tres pilares, el fomento del empleo de herramientas digitales para la comercialización, tanto de productos y servicios, de organizaciones que conforman la cadena de valor turística, como del destino turístico de manera global.

\vspace*{1.5em}


Estos cambios significativos, introducidos por las TIC´s vienen desempeñando una transformación, una dinamización de los procesos y un papel relevante en el sector turístico en la última década, pues reducen la duplicación de esfuerzos en la organización, aumentan la velocidad y la fiabilidad de las transacciones, mejoran la comunicación con los clientes y terceros, generan ganancias de eficiencia en la gestión, facilitan el acceso a la información interna y externa, contribuyen al reclutamiento y selección de empleados, mejoran el proceso interno.

\vspace*{1.5em}


El desarrollo de soluciones tecnológicas para la promoción turística del centro histórico de la ciudad de Matanzas, aún no ha sido implementado en profundidad. La ciudad se basa en sitios y páginas web para brindar información detallada a los usuarios sobre los distintos recursos y atractivos turísticos, así como edificaciones no turísticas; mientras que destinos europeos, asiáticos y latinoamericanos; diseñan e implementan herramientas de multimedia e interactivas para captar el interés de sus principales mercados.

\vspace*{1.5em}

Los visitantes internacionales que llegan a la ciudad de Matanzas, en su mayoría son procedentes de excursiones de las distintas agencias de viajes receptivas que existen en el destino. Por tanto, son visitas en distintos idiomas, guiadas por los profesionales que laboran en estas empresas intermediarias, pues la mayoría de estos visitantes no poseen vasta información sobre las distintas edificaciones que conforman la ciudad y tampoco cuentan con una solución tecnológica que le ofrezca información sobre la ciudad por lo que dependen exclusivamente de la experiencia y intereses de los guías turísticos. Esta situación trae como consecuencias negativas:

\begin{itemize}
	\item La no correcta promoción de toda la información relevante de ese destino lo cual repercute de forma negativa en una rápida apropiación de los contenidos que se ofrecen in situ, lo que repercute en una excelente interpretación del patrimonio cultural y natural.
	\item La no existencia de nuevas vías de obtener datos sobre determinada temática, lugar o mercado, apoyado sobre un sistema de recomendación provee a los consumidores turísticos mayor facilidad en la selección de determinado producto o servicio.
	
\end{itemize}




Teniendo en cuenta la situación planteada con anterioridad se define el siguiente \textbf{problema de investigación}: ¿Cómo desarrollar un sistema informático o software que contribuya en el valor comercial de la oferta turística del destino ciudad Matanzas.?

Planteándose como \textbf{hipótesis}:Mediante el uso de las herramientas y tecnologías actuales, es posible desarrollar un sistema que contribuya en el valor comercial de la oferta turística del destino ciudad Matanzas.

Para dar solución a este problema se asume como \textbf{objeto de estudio}:La puesta en valor comercial de la oferta turística.

En concordancia con lo anterior se propone como \textbf{objetivo general}: Desarrollar un sistema informático que permita aumentar los grados de la puesta en valor comercial de la oferta turística del destino ciudad Matanzas.

Del cual, se desagrega en \textbf{objetivos específicos}: Establecer un marco teórico referencial sobre la realidad aumentada y sus aplicaciones en la actividad turística, proponer un procedimiento para el diseño de una solución tecnológica basada en realidad aumentada que potencie la puesta en valor comercial de la oferta turística del destino Matanzas y desplegar el procedimiento propuesto en un objeto de estudio práctico

Enmarcado en el \textbf{campo de acción}: La puesta en valor comercial de la oferta turística del destino ciudad Matanzas.

\subsection*{Posibles resultados:}

\begin{enumerate}
	\item Sistema para potenciar la puesta en valor comercial de la oferta turística del destino ciudad Matanzas con el uso de la realidad aumentada.
	\item Mecanismo para la generación digital y formato duro de la puesta en valor comercial de la oferta turística del destino ciudad Matanzas.
	\item Herramienta de apoyo para el análisis y toma decisiones de los directivos y especialistas que se encargan de la puesta en valor comercial de la oferta turística del destino ciudad Matanzas.
\end{enumerate}

Para dar cumplimiento a los objetivos de esta investigación se definieron las siguientes \textbf{tareas investigativas}:

\begin{enumerate}
	\item Elaboración del marco teórico de la investigación a través del estudio del estado del arte que existe actualmente sobre el tema. 
	\item Identificación de los principales elementos que componen la puesta en valor comercial de la oferta turística.
	\item Identificación de los principales elementos que componen la puesta en valor comercial de la oferta turística ciudad Matanzas.
	\item Caracterización de los principales elementos que componen la puesta en valor comercial de la oferta turística.
	\item Caracterización de los principales elementos que componen la puesta en valor comercial de la oferta turística ciudad Matanzas.
	\item Realización del levantamiento de requisitos funcionales y no funcionales.
	\item Implementación del sistema que brinde solución al problema planteado.
	\item Realización de pruebas para validar el cumplimiento de los requerimientos.
\end{enumerate}

Durante la investigación se llevan a cabo varios métodos y técnicas en la búsqueda y procesamiento de la información como son:
Métodos Teóricos

\begin{itemize}
	\item \textbf{Analítico-sintético:} Para el estudio de los conceptos vinculados en los sistemas informáticos para la gestión de recursos que componen la oferta turística, y para el análisis de la documentación necesaria, permitiendo así, un mejor entendimiento del problema a resolver y realizar la extracción de los elementos más importantes para el desarrollo del trabajo.
	
	\item \textbf{Histórico-lógico:} Para realizar un análisis de las soluciones similares y las tendencias actuales en los sistemas informáticos enfocados en la gestión de recursos que componen la oferta turística.
	
	\item \textbf{Inductivo-Deductivo:} Para realizar el estudio de las principales herramientas existentes para el desarrollo de los sistemas informáticos para la gestión de recursos que componen la oferta turística y según las características de las mismas, se definieron las cualidades que debe cumplir el sistema que se propone en el presente trabajo de diploma. 
	
	\item \textbf{Modelación:} Para representar gráficamente conceptos y procesos con la finalidad de un mejor entendimiento de la solución que se propone.  
\end{itemize}

Métodos Empíricos

\begin{itemize}
	\item \textbf{Observación científica:} Para conocer el funcionamiento actual del proceso de gestión de recursos que componen la oferta turística del destino ciudad Matanzas, lo que permitió detectar las dificultades existentes en dicho proceso.
	\item \textbf{Consulta bibliográfica:} Para consultar y analizar las fuentes de información relacionadas con los diversos tipos de sistemas informáticos para la gestión de recursos que componen la oferta turística.
	
	\item \textbf{Generalización:} Permite sistematizar en cada capítulo de la investigación los aspectos más significativos y llegar a conclusiones más objetivas y explícitas.
\end{itemize}

Técnicas para la obtención de información:

\begin{itemize}
	\item \textbf{La entrevista:} Para la realización de encuentros planificados con los especialistas del proceso para obtener la información necesaria que será utilizada para el desarrollo del trabajo, posibilitando una buena comunicación y una participación activa y directa entre el equipo de desarrollo y el cliente.
	
\end{itemize}

La estructura del documento se resume en los siguientes acápites:

\paragraph*{Capítulo 1. Fundamentación teórica:} Abarca la elaboración del marco teórico de la investigación. De igual manera se analizan aplicaciones de gestión de recursos. Además se exponen las características principales de estos sistemas. Se analizan las principales tendencias, tecnologías, metodologías y softwares utilizados en la actualidad para el desarrollo de aplicaciones de gestión de recursos . A su vez se analizan y se fundamenta la selección de estas para el desarrollo de la solución propuesta. 


\paragraph*{Capítulo 2. Análisis y diseño del sistema:} Se reflejan las actividades realizadas en los procesos de análisis y diseño de la solución propuesta; proceso que será guiado por la metodología de desarrollo seleccionada. En el mismo se realiza el modelo de dominio donde se describen las entidades que intervienen con el objetivo de facilitar la comprensión de los principales conceptos que se utilizarán en el proceso de negocio identificado. Se exponen los artefactos más importantes que describen el flujo normal de eventos que ocurren en el sistema, se realiza una descripción de la solución propuesta, planteándose los requisitos funcionales y no funcionales. Se define la arquitectura que tendrá la solución propuesta.

\paragraph*{Capítulo 3. Implementación y pruebas:} En este capítulo se describe la fase de implementación del sistema, según la metodología propuesta. Se realizan además una serie de pruebas que permiten validar el correcto funcionamiento de  la solución, verificándose así, que el mismo cumple con todos los requerimientos y exigencias del cliente.

\vspace*{1.5em}

Finalmente, se presentan las conclusiones y las recomendaciones de la investigación para dejar el camino abierto a futuros estudios relacionados con el tema abordado.
De igual forma, quedan recogidas las bibliografías y anexos que fueron utilizados y conformados respectivamente para el desarrollo de la solución.