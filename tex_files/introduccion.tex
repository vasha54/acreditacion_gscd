La carrera Gestión Sociocultural para el Desarrollo (GSD) es la continuidad y resultado del perfeccionamiento curricular de la carrera Estudios Socioculturales, que inició en el curso 1999 – 2000 en la Universidad de Cienfuegos \emph{Carlos Rafael Rodríguez}, con el objetivo de potenciar el estudio de las ciencias sociales y humanísticas, y junto a ello formar un profesional con diversas habilidades, para impulsar iniciativas y proyectos que favorecieran cambios desde lo sociocultural, así como la elevación de la calidad de vida y el protagonismo de la población en dicha transformación.

La carrera Estudios Socioculturales abre por primera vez en la Universidad de Matanzas en el año 2001, ante la aceptación y perspectivas se fue instaurando de manera progresiva en otras universidades, en varias modalidades de estudio, con una extensión sin precedente para ninguna otra carrera universitaria hasta ese momento.

Hasta la fecha la carrera ha transitado por diferentes planes de estudio, orientados y dirigidos por la Comisión Nacional de Carrera, entre ellos están el Plan de Estudio C, Plan de Estudio C (perfeccionado), Plan de Estudios D y finalmente Plan de Estudios E. Este último trajo implícito el cambio de nombre de la carrera a Gestión Sociocultural para el Desarrollo, una profesión con una visión holística, consecuente con el proyecto social cubano que asume el desarrollo como proceso multilateral dirigido a construir un socialismo próspero y sostenible, y por ello mismo esencialmente sociocultural, lo que requiere una atención profesional sostenida sobre los aspectos culturales y espirituales en su sentido amplio. 

En la Universidad de Matanzas la carrera abre sus puertas en el curso escolar 2016-2017, y desde la primera graduación en julio de 2021 hasta la fecha se han graduado 36 profesionales en la modalidad de Curso Diurno, pertenecientes todos a la provincia de Matanzas, los que han sido ubicados en entidades pertenecientes a Organismos de la Administración Central del Estado. 

Este Plan de Estudios E se encuentra estructurado en 10 disciplinas a cursar en cuatro años de carrera en la modalidad de Curso Diurno. El claustro está integrado por 46 profesores, 19 Doctores en Ciencias (41.30\%), 26 Máster en Ciencias y solo un profesor es Licenciado (2.17\%), el cual se encuentra próximo a su predefensa doctoral en el Programa de Ciencias de la Educación. Los profesores cuentan con gran experiencia en la formación académica, de los cuales 33 cuentan con categoría docente principal de Titular y Auxiliar, lo que representa un 71.73\%. Una particularidad importante es que se cuenta en el Departamento Docente con graduados de los tres planes de estudios de la carrera, lo que garantiza una visión integral de la misma, reforzando de esta manera el trabajo con los modos y esferas de actuación profesional.

La superación profesional del claustro es una tarea constante en la carrera, los profesores en formación están todos en programas doctorales y de maestrías, destacando entre ellos el Programa Doctoral en Ciencias de la Educación y la Maestría en Estudios Sociales y Comunitarios (programa de la Facultad de Ciencias Sociales y Humanidades). Otros estudios doctorales están dirigidos hacia las áreas de Ciencias Económicas, Ciencias Históricas, Ciencias Sociológicas y Ciencias de la Comunicación. El desarrollo científico del claustro se enaltece por la labor investigativa en variados proyectos nacionales, sectoriales y territoriales de gran importancia, destacando su presencia en eventos científicos de elevado prestigio en Cuba y en el extranjero.

Desde la carrera se realiza una labor educativa que garantiza la formación integral de los estudiantes, a partir del Modelo del Profesional y el diseño del objetivo integrador de cada año académico. Los colectivos pedagógicos están conducidos por profesores de experiencia y categoría docente superior, enfatizando en el claustro del primer año. El diseño de la estrategia educativa de la carrera y de los colectivos de años garantizan el protagonismo de los estudiantes en su formación, con un elevado compromiso con la transformación social del territorio.

Actualmente la carrera posee la \underline{Categoría de Avalada (Calificada)} con fecha junio de 2017, dada por la Junta de Acreditación Nacional del Ministerio de Educación Superior en Cuba. Desde ese propio año se trabaja con el propósito de perfeccionar la labor educativa, científica y metodológica de profesores y estudiantes, así como en las debilidades detectadas en ese momento para superar esa categoría. 

Para organizar el proceso, y garantizar los resultados esperados, se crearon cinco equipos de trabajos dirigidos por la Coordinadora de Carrera, M. Sc. Yinela Castillo Lozano, integrados por los siguientes profesionales:

\vspace*{2cm} 


\begin{longtable}{|c|c|c|}
	
		\endfirsthead
	
	\mc{3}{>{}c}{{} Continuación de la página anterior }\\ 
	
	\endhead
	
	\hline
	\underline{\textbf{VARIABLES}}	& \underline{\textbf{RESPONSABLE}}  &  \underline{\textbf{MIEMBROS}} \\
	\hline
	Variable 1 & M. Sc. Yara Antonia Alfonso Cobas & Lic. Lisandra Lantigua Hernández \\


	&  & Lic. Enmanuel Castellanos Tápanes \\
	\hline
Variable 2	& Dr. C. Rosa Elvira Alfonso Ramos  & M. Sc. Gerardo Antonio Mier Daubar  \\
	
	&  & Dr. C. Ana Gloria Peñate Villasante \\
	
	&  & Lic. Arlette Ortega Arencibia \\
	\hline
	Variable 3 & M. Sc. Yailet Morales Delgado & Lic. Guillermo Alfredo Jiménez Pérez \\
	\hline
	Variable 4 & M. Sc. Pedro Antonio Busot Silva & Lic. Nicolás de Jesús Alfonso Alba \\
	\hline
	Variable 5 & M. Sc. Yinela Castillo Lozano & Lic. Arllete Berenice Barroso Quesada \\
	\hline
\end{longtable}




Los resultados de la autoevaluación exponen que la carrera Gestión Sociocultural para el Desarrollo en la Universidad de Matanzas tiene un trabajo de perfeccionamiento continuo y un importante nivel de desempeño, en correspondencia con las variables, los indicadores y los criterios de evaluación. A continuación, se exponen los resultados del proceso de autoevaluación, teniendo en cuenta los ítems definidos en el Modelo SEA-CU, sobre el Sistema de Evaluación y Acreditación de Carreras Universitarias.