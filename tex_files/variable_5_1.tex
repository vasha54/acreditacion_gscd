La estructura curricular de la carrera se encuentra en coherencia con lo concebido en el Modelo del Profesional y en correspondencia con las demandas del territorio, dirigidos en su esencia a potenciar el desarrollo humano, incidiendo en el enriquecimiento espiritual, fortalecimiento de la identidad cultural y nacional, la calidad de la vida colectiva y capacidad de participación de la población en el desarrollo social en sentido general.

El currículo asegura el trabajo con los modos y esferas de actuación, con un carácter sistémico, integrador, flexible y en función de la unidad entre la educación y la instrucción. En el currículo base se incluyen disciplinas de carácter específico (Gestión Sociocultural, Metodología Social, Desarrollo y Políticas Sociales e Historia Cultural y Pensamiento Social) y de carácter general con contenido filosófico, histórico y otras con aportes necesarios de las ciencias en particular y que posibilitan contenidos indispensables para desempeñarse como gestores socioculturales. El currículo propio está diseñado en función de las necesidades y potencialidades del territorio, teniendo en cuenta para ello las características del claustro. En este particular se cuenta con cinco asignaturas propias. El currículo optativo / electivo está pensado para el desarrollo de una cultura general integral en los estudiantes, teniendo en cuenta complementa la formación profesional. Se cuenta con el diseño de cinco asignaturas entre optativas y electivas. De manera general, el diseño curricular permitirá el actuar de forma progresiva en la medida que se logra una formación continua, respondiendo a la realidad sociocultural de los espacios donde este gestor incida. 

Los objetivos del Modelo del Profesional se concretan en las disciplinas, colectivos de años y carrera, en correspondencia con los modos y esferas de actuación, con una labor formativa consecuente con el perfil profesional y las estrategias curriculares. Los objetivos y contenidos de las disciplinas y las asignaturas dan respuesta a los requerimientos de la carrera y están en correspondencia con el objetivo integrador de cada año académico. La Disciplina Principal Integradora (Gestión Sociocultural) tiene como punto de referencia el objeto de trabajo del egresado, pues integra, controla y responde por el dominio y desarrollo de los modos y esferas de actuación de la profesión. Para ello cuenta con la presencia de asignaturas en todos los años académicos y con una estrategia orientada al cumplimiento y desarrollo de la práctica laboral investigativa. Su propósito principal es la formación de la habilidad rectora de la carrera (la gestión sociocultural), que su formación y desarrollo transversaliza la carrera con un vínculo permanente entre la teoría y la práctica.

Los colectivos pedagógicos en todos los niveles están dirigidos por profesores con elevada experiencia en la función que desempañan y en su 100\% con categoría docente superior. En la tabla se refleja la composición de los colectivos pedagógicos de carrera, año y disciplinas en correspondencia con las exigencias y estándares de calidad establecidos por el SEA-CU:

{%

	\newcommand{\mc}[3]{\multicolumn{#1}{#2}{#3}}

\begin{longtable}{|c|c|c|p{4.5cm}|}
	
		\endfirsthead
	
	\mc{2}{>{}c}{\tablename\ \thetable{} Continuación de la página anterior }\\ 
	
	\endhead
	
	
	
	\hline
\underline{\textbf{Colectivo Carrera}}	& \underline{\textbf{Profesor}} &  \underline{\textbf{Categoría Docente}} & \underline{\textbf{Categoría Científica}}  \\
	\cline{2-4}
	& \parbox[t]{3cm}{Yinela Castillo Lozano} & Profesora Auxiliar & Máster en Estudios Históricos y Antropología Sociocultural Cubana
	Doctaranda en Programa Doctoral de Historia  \\
	
	
	\hline
	
	
	\underline{\textbf{Colectivo de Año}} & \underline{\textbf{Profesor}} & \underline{\textbf{Categoría Docente}} & \underline{\textbf{Categoría Científica}}  \\
		\hline 
	\mc{4}{|>{}r|}{Continúa en la siguiente página }\\  
	\hline
	1er Año & \parbox[t]{3cm}{Yailet Morales Delgado} & Profesora Auxiliar  & Máster en Educación Superior
	Doctaranda en Programa Doctoral Ciencias de la Educación  \\
	\hline
	2do Año & \parbox[t]{3cm}{Gerardo Antonio Mier Daubar} & \parbox[t]{3.5cm}{Profesor Asistente (Con 43 años de experiencia en educación y 20 años de experiencia específicamente en la Educación Superior)} & Máster en Ciencias de la Educación Superior \\
	\hline
	3er Año & \parbox[t]{3cm}{Yara Antonia Alfonso Cobas} & Profesora Auxiliar & Máster en Desarrollo Comunitario  \\
	\hline
	4to Año & \parbox[t]{3cm}{Ana Gloria Peñate Villasante}  & Profesora Titular  & Doctora en Ciencias de la Educación \\
	\hline
	\underline{\textbf{Colectivo Disciplina}} & \underline{\textbf{Profesor}}  & \underline{\textbf{Categoría Docente}}  & \underline{\textbf{Categoría Científica}}  \\
	\hline
	Gestión Sociocultural & \parbox[t]{3cm}{Yailet Morales Delgado} & Profesora Auxiliar & Máster en Educación Superior
	Doctaranda en Programa Doctoral Ciencias de la Educación \\
	\hline
	Metodología Social & \parbox[t]{3cm}{Odalis Alberto Santana} & Profesor Titular & Doctora en Ciencias de la Educación \\
	\hline
	\parbox[t]{3cm}{Desarrollo y Políticas Sociales} & \parbox[t]{3cm}{Yara Antonia Alfonso Cobas}  & Profesora Auxiliar  & Máster en Desarrollo Comunitario  \\
	\hline
	\parbox[t]{3cm}{Historia Cultural y Pensamiento Social} & \parbox[t]{3cm}{Silvia Teresita Hernández Godoy} & Profesora Titular & Doctora en Ciencias Históricas \\
		\hline 
	\mc{4}{|>{}r|}{Continúa en la siguiente página }\\  
	\hline
	\parbox[t]{3cm}{Marxismo - Leninismo}& \parbox[t]{3cm}{María Felicia Ibáñez Matienzo} & Profesora Auxiliar & Máster en Desarrollo Comunitario \\
	\hline
	Historia de Cuba & \parbox[t]{3cm}{Oscar Andrés Piñeira Hernández} & Profesor Titular & Doctor en Ciencias Históricas \\
	\hline
	Computación & \parbox[t]{3cm}{Lázaro Tió Torriente}  & Profesor Titular & Doctor en Ciencias de la Educación \\
	\hline
	\parbox[t]{3cm}{Estudios de la Lengua Española } & \parbox[t]{3cm}{Rosa Elvira Alfonso Ramos} & Profesora Titular & Doctora en Ciencias Pedagógicas  \\
	\hline
	\parbox[t]{3cm}{Preparación para la Defensa} & \parbox[t]{3cm}{Luis Orlando Milián Zambrana} & Profesor Asistente & Máster en Estudios Sociales y Comunitarios \\
	\hline
	Educación Física & \parbox[t]{3cm}{Ángel Fidel Llanos González} & Profesor Asistente & Máster en Ciencias de la Educación Superior \\
	\hline
	\caption{Colectivos pedagógicos de la carrera en la Universidad de Matanzas (Elaboración propia)} 
	\label{tableclaustro}
\end{longtable}

}%

El trabajo metodológico de la carrera en los diferentes niveles organizativos se articula con las prioridades de la Universidad y de la Facultad. Es un proceso sistemático que potencia las relaciones inter, intra y transdiciplinarias en torno a la Disciplina Principal Integradora y el desarrollo de la habilidad profesional integradora, lo que se expresa en el nivel de actualización constante de las asignaturas que conforman el currículo base, propio y optativo / electivo de la carrera, en torno al cumplimiento del objetivo integrador del año académico y en relación con los trabajos de curso y la práctica laboral investigativa. El trabajo metodológico sistemático favorece el proceso docente educativo en relación a:

\begin{itemize}
	\item Las investigaciones pedagógicas realizadas en las disciplinas han estado enfocadas al perfeccionamiento de las asignaturas, la interdisciplinariedad, desarrollo de valores, elaboración de materiales didácticos, diseños de programas, a partir de la experiencia acumulada.
	\item Se encuentran correctamente estructurados los programas de asignaturas en función del trabajo metodológico que se realiza a nivel de las disciplinas y carrera.
	\item En los programas de asignaturas aparece declarado el sistema de conocimientos, habilidades y valores a potenciar en los estudiantes, además de las estrategias curriculares y la forma en que se implementan.
	\item Las evaluaciones frecuentes, parciales y finales garantizan la medición del cumplimiento de los objetivos declarados por asignaturas y disciplinas en relación con el objetivo integrador del año académico.
	\item En cada una de las clases y actividades se emplean de forma efectiva los métodos, formas organizativas, medios y sistemas de evaluación.
	\item Se exige la utilización de las TICs en las asignaturas, trabajos de curso o diploma.
	\item Se realiza un control de la organización y proyección del colectivo para el cumplimiento con calidad del objetivo integrador del año y la integración de los aspectos educativos e instructivos con un enfoque interdisciplinario.
	\item El ejercicio de culminación de estudios se realiza en correspondencia con las diferentes líneas de investigación de la carrera y de las instituciones, organismos, empresas y organizaciones del territorio, teniendo en cuenta para ello la ubicación laboral anticipada y la participación en proyectos de investigación.
\end{itemize}