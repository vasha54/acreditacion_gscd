La carrera cuenta con un total de 46 profesores con diferentes categorías científicas y docentes. Para ello se ha trabajado por una estrategia de formación profesional, llevada a cabo por el Departamento Docente, la Facultad de Ciencias Sociales y Humanidades y la Universidad de Matanzas, que a corto plazo ha propiciado saldos positivos.

En la estructura del claustro, el 97.8\% del total de profesores son doctores y máster. Un docente no es doctor ni máster, pero se encuentra en el programa de Formación Doctoral, con proyección a defender en el curso 2023.

En cuanto a la estructura del claustro por categoría docente, puede apreciarse en la Tabla 2.2.1 que el 71.7\% de los profesores posee las categorías docentes principales de Titular y Auxiliar.

\begin{longtable}{|p{12cm}|c|c|}
	\hline
	\multicolumn{1}{|c|}{ \underline{\textbf{Categorías Docentes}}} & \underline{\textbf{Total}} & \underline{\textbf{\%}} \\ \hline
	Profesor Categoría Docente Superior (Titular y Auxiliar) &33  & 71.73 \\ \hline
	Profesor Asistente & 13 & 28.26 \\ \hline
	\caption{Estructura del claustro por categoría docente (Elaboración propia)}
\end{longtable}

Los docentes con categoría superiores son los que ocupan las funciones de Profesores Principales de Año (PPA) y coordinan las disciplinas de la carrera, aunque se preparan en estas responsabilidades, jóvenes profesores que están en el proceso de cambio de categoría docente y científica. Con ello se garantiza la realización efectiva del trabajo educativo, metodológico y político-ideológico en estos espacios, además de la sostenibilidad del claustro.

Los profesores con menor categoría docente y científica, en particular los más jóvenes, son asesorados en los colectivos metodológicos de cada disciplina, con resultados satisfactorios.