La actividad investigativa laboral de los estudiantes se desarrolla con calidad y en correspondencia con las líneas y proyectos de investigación, lo que contribuye a la formación y desarrollo de sus modos y esferas de actuación. Desde la Disciplina Principal Integradora se coordina la práctica laboral investigativa, en correspondencia con las necesidades de las instituciones del territorio y el objetivo integrador del año académico. Se realizan asociadas a las entidades laborales de base y unidades docentes, contando en cada caso con un tutor al frente de su formación.

En los últimos años se ha fortalecido y consolidado el trabajo de la carrera con las unidades docentes y entidades laborales de base. Los directivos de la carrera participan en actividades de dichas entidades y viceversa, en función de lograr un mayor vínculo para la inclusión de estudiantes en las prácticas laborales investigativa y su atención particularizada.

La actividad investigativa laboral, se ha fortalecido con la integración de estudiantes a las sociedades científicas estudiantiles y proyectos de investigación, con estrecha relación con el posgrado. Actualmente el 100\% de los estudiantes están vinculados a proyectos y muestran resultados satisfactorios en la participación en eventos científicos, talleres, sesiones científicas, jornadas científicas estudiantiles.

Una labor importante para el éxito de las prácticas laborales y la investigación lo constituyen las unidades docentes de la carrera, constituidas en entidades laborales de prestigio profesional con los requisitos necesarios para la formación de los modos y esferas de actuación de la profesión. Estas unidades son: el Departamento de Investigación y Desarrollo de la Dirección Provincial de Cultura (Unidad de Desarrollo e Innovación), el CITMA, el CIGET el Centro Provincial de Patrimonio Cultural, el Archivo Histórico Provincial, el Ministerio de Trabajo y Seguridad Social. Algunos profesionales de estas unidades docentes son tutores para la práctica laboral investigativa, las investigaciones y trabajos de diploma, además que son profesores del claustro, lo que demuestra la vinculación de la carrera con profesionales del territorio.

Se muestra satisfacción por parte de los estudiantes con el trabajo realizado por estas instituciones y la atención que le brindan en investigaciones y durante la práctica laboral investigativa.