La eficiencia académica y la de ciclo tienden a disminuir. Al analizar las causas se debe tener en cuenta que en el Excel se introducen en los cursos 2017-2018, 2018-2019 y 2019-2020, las evaluaciones de primero a quinto año académico, pues aún egresan estudiantes del Plan de Estudio D (Licenciado en Estudios Socioculturales), cuya formación profesional es de cinco cursos. Esto trae como consecuencia que en el curso 2019-2020, el Excel solo reconoce a los 10 graduados de quinto año (Plan de Estudio D) y no a los 12 estudiantes que culminan sus estudios en cuarto año (Plan de Estudio E). En correspondencia con lo explicado, en los cursos 2021 y 2022 el quinto año aparece vacío. Este pudiera ser uno de los motivos por los cuales se vea afectada la eficiencia académica y la de ciclo.

De igual modo, se debe considerar que la matrícula de la carrera por años es baja, por lo que cualquier incidencia afecta este aspecto, como es el caso de los cambios de modalidad al Curso Por Encuentro (CPE) y las licencias de matrícula (LM).
  
La carrera posee una estrategia para mantener y mejorar los índices de permanencia, rendimiento y egreso, con énfasis en el primer año. Esta estrategia se gestiona y desarrolla a partir de diagnósticos realizados por colectivos de año (psicopedagógicos y por asignaturas), con acciones propuestas en las estrategias educativas, la labor de extensión desde la propia carrera, el desarrollo de cada una de las asignaturas, así como del seguimiento a la práctica laboral investigativa y la culminación de estudios. Dicha estrategia se controla por el colectivo de carrera.

En cuatro de los cursos del quinquenio que se evalúa hay estudiantes que han obtenido Títulos de Oro, excepto, en el curso 2018-2019, como se observa a continuación  en la Tabla \ref{tablediplomaoro}:

	
\begin{longtable}{|c|p{7cm}|c|}
	
		\endfirsthead
	
	\mc{2}{>{}c}{\tablename\ \thetable{} Continuación de la página anterior }\\ 
	
	\endhead
	
	\hline
	\underline{\textbf{Curso}}& \multicolumn{1}{c|}{ \underline{\textbf{Nombres y Apellidos}}} & \underline{\textbf{Promedio}} \\
	\hline
2017-2018	&  Sully Rodríguez Reyes &  4.79\\
	\cline{2-3}
	& Jessica Cou Perera & 4.87 \\
	\hline
2018 -2019	& - & - \\
	\hline
2019-2020	&  Mauren Bello Ruíz & 4.85 \\
	\cline{2-3}
	& Claudia Acebo Suárez & 4.90\\
	\hline
2021	& Arletty González Torres & 4.78 \\
	\cline{2-3}
	& Ana Laura Mederos González & 4.75 \\
	\hline
2022	&  Evelyn Fernández Sánchez & 4.85 \\
	\hline
	\caption{Títulos de Oro (Elaboración propia; Fuente: Documentos de Secretaría Docente FCSH)}
	\label{tablediplomaoro}
\end{longtable}

El perfeccionamiento de la labor en los aspectos que tributan a esta variable ha estado vinculado a la evaluación sistemática y crítica de los resultados alcanzados en cada curso, las exigencias en el estudio independiente, la labor educativa, el trabajo político ideológico y el perfeccionamiento del trabajo metodológico.

En particular, en el sistema de evaluación del aprendizaje, se ha priorizado la aplicación correcta de las instrucciones y reglamentos. Unido a ello la superación didáctico-metodológica del claustro de forma sistemática.
