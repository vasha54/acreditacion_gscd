La eficiencia académica muestra la inestabilidad en el rendimiento por permanencia en los últimos 5 años. Teniendo en cuenta que la matrícula de la carrera por años es baja, por lo que cualquier incidencia afecta este aspecto. La eficiencia académica, la vertical y la de ciclo en el período que se evalúa se ve afectada por las siguientes causas, (se cita como ejemplo el curso 2022, por ser el año base para el proceso de autoevaluación): 1er año con 1 cambio de modalidad al Curso Por Encuentro (CPE) y 1 licencia de matrícula (LM); 2do año con 2 cambios de modalidad a CPE y 4 causan baja al finalizar el curso; 3er año con 1 baja al finalizar el curso; y 4to año con 1 baja al finalizar el curso, que no se presentó a culminación de estudio. 

La carrera posee una estrategia para mantener y mejorar los índices de permanencia, rendimiento y egreso, con énfasis en el primer año. Esta estrategia se gestiona y desarrolla a partir de diagnósticos realizados por colectivos de año (psicopedagógicos y por asignaturas), con acciones propuestas en las estrategias educativas, la labor de extensión desde la propia carrera, el desarrollo de cada una de las asignaturas, así como del seguimiento a la práctica laboral investigativa y la culminación de estudios. Dicha estrategia se controla por el colectivo de carrera.

En cuatro de los cursos del quinquenio que se evalúa hay estudiantes que han obtenido Títulos de Oro, excepto, en el curso 2018-2019, como se observa a continuación en el \hyperref[tablediplomaoro]{\textcolor{blue}{ siguiente \tablename}}:

	
\begin{longtable}{|c|c|c|}
	
		\endfirsthead
	
	\mc{2}{>{}c}{\tablename\ \thetable{} Continuación de la página anterior }\\ 
	
	\endhead
	
	\hline
	\underline{\textbf{Curso}}&  \underline{\textbf{Nombres y Apellidos}} & \underline{\textbf{Promedio}} \\
	\hline
2017-2018	&  Sully Rodríguez Reyes &  4.79\\
	\cline{2-3}
	& Jessica Cou Perera & 4.87 \\
	\hline
2018 -2019	& - & - \\
	\hline
2019-2020	&  Mauren Bello Ruíz & 4.85 \\
	\cline{2-3}
	& Claudia Acebo Suárez & 4.90\\
	\hline
2021	& Arletty González Torres & 4,78 \\
	\cline{2-3}
	& Ana Laura Mederos González & 4.75 \\
	\hline
2022	&  Evelyn Fernández Sánchez & 4.85 \\
	\hline
	\caption{Títulos de Oro (Elaboración propia; Fuente: Documentos de Secretaría Docente FCSH)}
	\label{tablediplomaoro}
\end{longtable}

El perfeccionamiento de la labor en los aspectos que tributan a esta variable ha estado vinculado a la evaluación sistemática y crítica de los resultados alcanzados en cada curso, las exigencias en el estudio independiente, la labor educativa, el trabajo político ideológico y el perfeccionamiento del trabajo metodológico.

En particular, en el sistema de evaluación del aprendizaje, se ha priorizado la aplicación correcta de las instrucciones y reglamentos. Unido a ello la superación didáctico – metodológica del claustro de forma sistemática.
