El personal no docente y administrativo cuenta con una elevada preparación profesional, lo que ha permitido su adecuado desempeño laboral. Con su sentido de pertenencia y responsabilidad, favorecen al logro de los objetivos previstos en la formación integral del estudiante. 

Las trabajadoras de la secretaría docente por su compromiso y confiabilidad contribuyen al perfeccionamiento de la documentación de la carrera tanto en el pregrado como en el posgrado y forma parte de este colectivo una graduada de la carrera Estudios Socioculturales.

Otro personal indispensable por el perfil de la carrera son los que se encuentran en las unidades docentes, quienes se encargan de la tutoría de los estudiantes de la carrera durante su período de práctica laboral investigativa, entre ellos: Departamento de Investigación y Desarrollo de la Dirección Provincial de Cultura (Unidad de Desarrollo e Innovación), el CITMA, el CIGET, el Centro Provincial de Patrimonio Cultural, el Archivo Histórico Provincial y el Ministerio de Trabajo y Seguridad Social.

También contribuyen al desarrollo de la formación profesional de los estudiantes de la carrera diversos grupos de trabajadores de la institución entre los que se destacan: Departamento de Recursos para el Aprendizaje, Departamento de Informatización, la Dirección de Extensión Universitaria y el personal de la Residencia Estudiantil.

De forma general, el personal de apoyo a la docencia resulta vital para el logro de resultados positivos en el trabajo, lo que favorece el desempeño exitoso del proceso docente-educativo.\\

\textbf{Fortalezas:}

\begin{itemize}
	\setlength\itemsep{-0.5em}
	\item Elevada Influencia del claustro de profesores en la educación de los estudiantes, basada en el alto compromiso con nuestro proyecto social, con una visible formación político-ideológica. Con alto grado de actualización y prestigio en el proceso de formación del estudiante, destacándose su estabilidad, profesionalidad, sentido de pertenencia, ejemplaridad, ética, preparación política e integral, demostrada en los resultados positivos alcanzados, lo cual favorece el desempeño exitoso del proceso docente-educativo.
	\item Excelente predominio de liderazgo de profesores con categoría docente Titular y Auxiliar en la coordinación del trabajo metodológico, a nivel de departamento, carrera, colectivos de año y disciplinas. Con un elevado nivel científico del claustro de la carrera compuesto por 46 profesores, donde 33 ostentan categoría docente principal (71.73\%), 19 Doctores en Ciencias (41.30\%) y 26 Máster en Ciencias (56.52\%) y 1 licenciado en formación doctoral.
	\item Existe presencia en el claustro de profesores graduados de la carrera (de sus diferentes planes de estudio), lo que garantiza una visión holística de la misma; así como especialistas con experiencia y vinculados a instituciones culturales y científicas de la provincia.
	\item Elevada la participación de profesores del claustro como responsables de órganos, comisiones y procesos de alta significación social, en tareas que tributan a proyectos de investigación.
	\item Es significativo el apoyo del personal auxiliar en el proceso formativo del profesional.
	\item Se cuenta con sistema de postgrado propio que satisface la demanda de superación de los graduados de la carrera y perfiles afines, así como las demandas de superación del territorio.
\end{itemize}

\textbf{Debilidades:}

\begin{itemize}
	\setlength\itemsep{-0.5em}
	\item No se declaran
\end{itemize}