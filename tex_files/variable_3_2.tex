La Disciplina Principal Integradora (DPI) diseña ejercicios teniendo en cuenta el objetivo integrador de cada año académico, asumiendo como base los modos y esferas de actuación profesional, que incluyen no solo conocimientos, habilidades y valores, sino aportes a las estrategias curriculares que se trabajan en el año que se encuentra el estudiante. En correspondencia con los modos y esferas de actuación y con las habilidades a desarrollar en cada año, se realizan las prácticas laborales investigativas en unidades docentes y entidades laborales de base que responden a la modalidad de gestión que se esté trabajando.

Se realizan exámenes integradores en cada uno de los años, a partir del diseño de los trabajos de curso correspondientes a las asignaturas de la DPI, que responden al objetivo integrador del año y a potenciar los modos y esferas de actuación del profesional, cuyos resultados se muestran a continuación en la Tabla \ref{tableresultados}:

\begin{longtable}{|p{1cm}|c|c|c|c|c|c|c|c|c|c|}
	\endfirsthead
	
	\mc{11}{>{}c}{\tablename\ \thetable{} Continuación de la página anterior }\\ 
	
	\endhead
	
	\hline
	& \mc{2}{>{}c|}{\underline{\textbf{Matrícula}} } & \mc{6}{>{}c|}{\underline{\textbf{Calificaciones \% }}} & \mc{2}{>{}c|}{\underline{\textbf{\% de aprobados }} }\\ \cline{4-9}
\underline{\textbf{Curso}}	& \mc{2}{>{}c|}{ } & \mc{2}{>{}c|}{ \underline{\textbf{3}} } & \mc{2}{>{}c|}{ \underline{\textbf{4}} } & \mc{2}{>{}c|}{\underline{\textbf{5}} } & \mc{2}{>{}c|}{\underline{\textbf{con 5ptos}}} \\
    \cline{2-11}
	& \underline{\textbf{1er}} & \underline{\textbf{2do}} & \underline{\textbf{1er}} & \underline{\textbf{2do}} & \underline{\textbf{1er}} & \underline{\textbf{2do}} &\underline{\textbf{1er}}  & \underline{\textbf{2do}} &\underline{\textbf{1er}} & \underline{\textbf{2do}} \\
	&\underline{\textbf{Período}}&\underline{\textbf{Período}}&\underline{\textbf{Período}}&\underline{\textbf{Período}}&\underline{\textbf{Período}}&\underline{\textbf{Período}}&\underline{\textbf{Período}}&\underline{\textbf{Período}}&\underline{\textbf{Período}}&\underline{\textbf{Período}}\\
	\hline
	2017-2018	&79&67&12&8&22&26&42&27&53.2\%&40.3\%\\
	\hline
	2018-2019	&81&69&18&6&35&26&26&31&32.1\%&45\%\\
	\hline
	2019-2020	&61&51&12&3&38&15&18&25&29.5\%&49\%\\
	\hline
	2021	&62&47&3& &16&10&40&32&64.5\%&68.1\%\\
	\hline
	2022	&50&40&3&4&14&4&32&27&64\%&67.5\%\\
	\hline
	\caption{Calidad de los resultados de ejercicios integradores por curso académico (Elaboración propia; Fuente: Actas de secretaría docente FCSH)}
	\label{tableresultados}
\end{longtable}

Para el análisis del Tabla \ref{tableresultados}, se debe tener en cuenta que la diferencia entre las cantidades de evaluaciones con el número de matrícula, está dada por los estudiantes no presentados a examen o que poseen licencia de matrícula y aparecen en acta. En el caso del 2do período, no se tiene en cuenta la matrícula del último curso de cada año académico porque es cuando presentan el ejercicio de culminación de estudio. Los estudiantes presentados a los ejercicios integradores representan un 100\% de aprobados. Los resultados más relevantes en \% con 5 puntos son los cursos 2021 y 2022.

Los trabajos de diploma han alcanzado calidad formal, científica y académica. Estos se han elaborado en correspondencia con las exigencias de la metodología de la investigación científica y se les orienta el empleo del idioma inglés en la elaboración de los resúmenes, palabras claves y utilización de fuentes en ese idioma.

A través de las diferentes formas de culminación de estudio realizadas en estos últimos cinco años (trabajos de diplomas, examen estatal, más los eximidos por resolución rectoral y los presentados en monografía, en casos excepcionales por la etapa de Covid 19) se muestra el incremento en la calidad, evidenciada en un 100\% de aprobados. Predominan las calificaciones de 5 puntos, siendo el curso 2022 el de mayor \%. 

\begin{longtable}{|c|c|c|c|c|c|c|c|}
		\endfirsthead
	
	\mc{8}{>{}c}{\tablename\ \thetable{} Continuación de la página anterior }\\ 
	
	\endhead
	\hline
	\underline{\textbf{Curso}} & \mc{3}{>{}c|}{\underline{\textbf{Matrícula}} }& \mc{3}{>{}c|}{\underline{\textbf{Calificaciones}} } & \underline{\textbf{\% con 5}}  \\
	\cline{5-7}
	& \mc{3}{>{}c|}{} & \underline{\textbf{3}} & \underline{\textbf{4}} & \underline{\textbf{5}} & \underline{\textbf{puntos}}  \\
	\hline
	2017-2018 & 12 & 8 & TD & & 1 & 7 & 66.7\% \\
	\cline{3-7}
	&  & 4 & EE & 3 & & 1 & \\
	\hline
	2018-2019 & 12 & 6 & TD & & 2 &4 & 75\% \\
	\cline{3-7}
	&  & 6&EE & &1 &4 & \\
	\hline
	2019-2020& 22 & 10 & TD & 1& 4& 5& 50\% \\
	\cline{3-7}
	&  & 3& EE& 1& 2& & \\
	\cline{3-7}
	&  & 6& E& & & 6& \\
	\cline{3-7}
	&  & 3& M& 1& 2& & \\
	\hline
	 2021&  15& 6 & TD& & 2& 4& 73.33\%\\
	\cline{3-7}
	&  & 1& EE& 1& & & \\
	\cline{3-7}
	&  & 5& E& & & 5& \\
	\cline{3-7}
	&  & 3& M& 1& & 2& \\
	\hline
	  2022&  9 & 6 & TD &  & 1 & 5 & 77.8\% \\
	\cline{3-7}
	&  & 3& EE& & 1& 2& \\
	\hline
	\caption{Resultados de los Trabajos de Diploma (TD) y Examen Estatal (EE), Eximido (E) y Monografía (M) (Elaboración propia; Fuente: Actas de Secretaría Docente FCSH)}
\end{longtable}


