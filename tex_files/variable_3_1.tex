Los estudiantes han tenido una activa participación como agentes del proceso formativo, lo que es apreciable a través del análisis de las estrategias educativas. Se realizan tareas en la labor política e ideológica, encaminadas a enriquecer la vida política y cultural universitaria, que se potencian desde la dimensión extensionista y sociopolítica de la estrategia educativa y desde las asignaturas del plan de estudio, evidenciadas en la formación profesional del período que se evalúa. Se destacan en la participación de actividades, en el marco de jornadas conmemorativas, actos políticos dedicados a efemérides relevantes de la historia y en eventos, como son:

 Inicio de la jornada por el Día de la Cultura Cubana y fundación de la ciudad de Matanzas, 7 de diciembre peregrinación y Operación Tributo, Bastión Estudiantil Universitario, participación en los debates del proyecto de la Constitución de la República en el año 2018, desfiles por el 1ero de mayo, marcha de las Antorchas. 

 La participación en actividades de reafirmación revolucionaria, tales como: en el 2018 la caminata desde la Escuela de Iniciación Deportiva (EIDE) hasta el Parque de la Libertad, por el aniversario de la muerte del Comandante en Jefe Fidel Castro Ruz. Actos realizados en el Parque de la Libertad (diciembre 2021).

De igual modo se participa en Días de la Carrera, Festival Universitario del Libro y la Lectura, Festival de Artistas Aficionados (se presenta el 20\% de la matrícula del curso 2022 al festival a nivel de facultad, distribuidos en las manifestaciones de artes visuales con mayor representación y en menor medida en literatura, audiovisual y música) llegando parte de ellos hasta nivel provincial y en espera de que se realice el Festival Nacional. 

Se participa en los Juegos Yumurinos, Festivales de la Clase (se presentan el 100\% de los alumnos ayudantes), en Congresos de la FEU (IX y X), Coloquios de la Décima y conferencias especializadas (Evento Teórico de la Fiesta de los Orígenes en Matanzas, 2022).

Es destacada también la participación en eventos y talleres, dentro de los que se encuentran:  Taller Cultura – Universidad; Taller Provincial de Educación Patriótico – Militar;  El Renacer del Movimiento de Izquierda en América Latina (Internacional); Taller Provincial de Impacto Económico y Social; Taller Provincial de Cultura, Género y Sociedad; Taller Fidel y el Socialismo en Cuba; Taller Científico Atenas 2018; II Encuentro de Ciencias Sociales y Estudios Socioculturales; Evento Patria, Símbolo e Identidad; Evento Puentes; eventos de desarrollo local; eventos de la Cátedra de Género, Cultura y Sociedad; talleres virtuales, fundamentalmente en el curso 2022, tales como: Taller Virtual Cultivando el Saber; Taller Virtual 8 Inocentes; Taller Virtual Celia y la Mujer Latinoamericana; Mella en las Luchas Estudiantiles Internacionales Actuales; Taller Virtual Historia Viva, entre otros.

Los estudiantes se integran a las actividades de las Cátedras Honoríficas y Multidisciplinarias (Cátedra de Arqueología y Cátedra Nicolás Guillén principalmente), desde la Cátedra de Arqueología realizan investigaciones que se concretarán como resultados de trabajos de diploma. Forman parte de proyectos de investigación y proyectos socioculturales, en los cuales han desarrollado una labor investigativa que culmina con la defensa de sus trabajos de diploma.

En el caso de los proyectos socioculturales, los cursos en los que más resaltó la participación de los estudiantes fueron 2017-2018, 2018-2019 y 2019-2020, en los proyectos A Moverse por el Cambio, Galería Joel Peláez y AfroAtenas. En los cursos 2021 y 2022 se logra mayor incorporación en los proyectos de investigación, resultados que se muestran en la tabla del Excel. Estos proyectos de investigación son: 

\begin{itemize}
	\setlength\itemsep{-0.5em}
	\item Proyecto Identidad y realidad cubana: estudio sociocultural del impacto de las transformaciones socioeconómicas en el centro de Cuba (PAP) (I+D+I) Programa Nacional de Ciencias Sociales y Humanísticas. Jefa de Proyecto: M. Sc. Soilen Cedeño Solis. 
	\item Proyecto Estudios socioculturales para el desarrollo sostenible: UM-DECORARTE (PNAP). Jefa de Proyecto: Dr. C. Rosa Elvira Alfonso Ramos.
	\item Proyecto VIDAS en el Rabí (PAPT). Jefe de Proyecto: Dr. C. Odalis Alberto Santana Territorial
	\item Proyecto Perfeccionamiento de la Gestión Universitaria (PNAP). Jefa de Proyecto: Dr. C. Lourdes Tarifa Lozano 
	\item Proyecto \emph{CCI \& CITYTOUR}:  la competencia comunicativa intercultural en el discurso de interpretación patrimonial para el desarrollo local del turismo de ciudad (PNAP). Jefe de Proyecto: Dr. C. Jorge Luis Rodríguez Morell 
	\item Gestión sociocultural para el desarrollo local en el Consejo Popular Matanzas Este (PAPT). Jefe de Proyecto: Dr. C. Ana Gloria Peñate Villasante
	\item Patrimonio cultural y formación: patrimonio cultural universitario (PCU), historia, educación patrimonial y desarrollo local. Jefa de Proyecto: Dr. C. Lissette Jiménez Sánchez.
\end{itemize}

Los estudiantes se encuentran integrados a las sociedades científicas estudiantiles, en las temáticas de: desarrollo local e intervención comunitaria, procesos culturales cubanos, extensión universitaria y gestión integral del patrimonio. 

En la formación de pregrado también han desarrollado tareas de impacto social en su interacción social con la comunidad universitaria y extrauniversitaria, en las que han potenciado la formación de sus modos y esferas de actuación y han contribuido a la transformación social, así como a su formación integral. Entre estas tareas se destacan: acciones recuperativas de eventos meteorológicos; gestión de las muestras y exposiciones que se presentan en la Galería Joel Peláez, ubicada en el departamento de Estudios Socioculturales; participación en INNOVAMATANZAS (2018) aplicando encuestas a la población y presentándose los resultados de esta tarea en el taller de cierre del evento. En el curso 2019-2020, se encuestaron a jóvenes matanceros para un estudio nacional sobre adolescencias y juventudes, coordinado por la UJC. En el curso 2022, los estudiantes de 4to año participaron en las actividades realizadas en la provincia de Matanzas por el Movimiento Sembrar Con Ciencia. 

Se destaca la participación en la organización, promoción y desarrollo de las actividades extensionistas de la Facultad de Ciencias Sociales y Humanidades (campaña Violencia Cero, Cátedra de Arqueología, Cátedra Nicolás Guillén). Las labores en apoyo a la COVID 19, donde estudiantes y profesores estuvieron incorporados a las pesquisas, a la zona roja, al trabajo en organopónicos y a otras actividades convocadas para el enfrentamiento a la pandemia, en colegios electorales, en actividades del Proyecto Galería Abierta, en las presentaciones de libros y en acciones comunitarias en circunscripciones, especialmente las acciones de apoyo comunitario integrado realizadas durante el curso 2021.

La carrera ha contado en sus diferentes cursos, con alumnos ayudantes que ofrecen apoyo en el trabajo docente educativo, formando parte de este movimiento en la Educación Superior, en la Enseñanza Media específicamente se han incorporado al Contingente Educando por Amor 

\begin{longtable}{|c|c|p{7cm}|}
		\endfirsthead
	
	\mc{3}{>{}c}{\tablename\ \thetable{} Continuación de la página anterior }\\ 
	
	\endhead
		\hline 
	\multicolumn{1}{|c}{\underline{\textbf{Curso}} } 
	& \multicolumn{1}{|c}{\underline{\textbf{Año Académico}}} 
	& \multicolumn{1}{|c|}{\underline{\textbf{Nombres y Apellidos}}}\\
	\hline 
	
	2017-2018 & 4to & Reynol Alfonso Rodríguez\\
	\cline{2-3}
	& 4to  & Mariem Cabrera Sánchez\\
	\cline{2-3}
	& 4to & Sulay Yinet Galbán Carrillo\\ 
	
	\cline{2-3}
	& 4to & Claudia Danay Hernández Arredondo\\
	\cline{2-3}
	& 4to & Yariany Mayor Sánchez\\
	\cline{2-3}
	& 4to & Lídice Moreno Pernas\\
	\cline{2-3}
	& 4to & Amhed Antuan Morgan Neninger\\
	\cline{2-3}
	& 4to & Liliana Naranjo Martínez\\
	\cline{2-3}
	& 4to & Susej Nieblas Santos\\
	\cline{2-3}
	& 4to & Yalenys Segura Arias\\
	\hline
	2018-2019 & 4to & Julio Carlos Acosta Risco\\
	\cline{2-3}
	& 4to & Maureen Bello Ruíz\\
	\cline{2-3}
	& 4to & Lauren Isel Herrera Pérez\\
	\cline{2-3}
	& 4to & Betty Rodríguez Orihuela	\\
	\cline{2-3}
	& 4to & Yuselmi Lázara Vázquez Herrera\\
	\cline{2-3}
	& 4to & David Alejandro Zulueta\\
	\cline{2-3}
	& 3ero& Etianys Alfonso Dueñas\\
	\hline
	2022 & 1ero & Aylín Sánchez Luis\\
	\cline{2-3}
	&  2do & Fernanda Pérez Gordillo\\
	\cline{2-3}
	& 2do & Mariangela Rufín Cao\\
	\hline
	\caption{Estudiantes vinculados al contingente Educando por Amor (Elaboración propia)}
\end{longtable}

\begin{longtable}{|c|p{7cm}|c|c|c|}
	\endfirsthead
	
	\mc{5}{>{}c}{\tablename\ \thetable{} Continuación de la página anterior }\\ 
	
	\endhead
	 \hline
	 \underline{\textbf{Curso}} & \multicolumn{1}{|c|}{\underline{\textbf{Nombres y Apellidos}}}& \underline{\textbf{Cantidad de Estudiante}} & \underline{\textbf{Matrícula}} & \underline{\textbf{Porciento}} \\
	 \hline
	2017-2018 & Leandro Larena Martínez & 11 & 79 & 14\% \\
	 \cline{2-2}
	& Marien Cabrera Sánchez & & & \\
	\cline{2-2}
	& Lisibet García Leyva & & & \\
	\cline{2-2}
	& Claudia Acebo Suárez & & & \\
	\cline{2-2}
	& Wendy Ros Villavicencio & & & \\
	\cline{2-2}
	& Dairys Cordero Tortoló & & & \\
	\cline{2-2}
	& Etianys Alfonso Dueñas & & & \\
	\cline{2-2}
	& Ann Sheyla Mirabal Navia & & & \\
	\cline{2-2}
	& Arlette Ortega Arencibia & & & \\
	\cline{2-2}
	& Daniela Daniel Gómez& & & \\
	\cline{2-2}
	& Mariam de la C. Cabrera Alarcón & & & \\
	\hline
	2018-2019& Ana Laura Mederos González & 16 & 81 & 20\% \\
	\cline{2-2}
	& Claudia Acebo Suárez & & & \\
	\cline{2-2}
	& Wendy Ros Villavicencio & & & \\
	\cline{2-2}
	& Yanay Fundora Plasencia & & & \\
	\cline{2-2}
	& Ana Laura Abreu Alfonso& & & \\
	\cline{2-2}
	& Ann Sheyla Mirabal Navia& & & \\
	\cline{2-2}
	& Arllete Barroso Quesada& & & \\
	\cline{2-2}
	& Lorena de la Caridad Gaskins Remond& & & \\
	\cline{2-2}
	& Arlette Ortega Arencibia & & & \\
	\cline{2-2}
	& Daniela Daniel Gómez& & & \\
	\cline{2-2}
	& Lisibet García Leyva& & & \\
    \cline{2-2}
	& Leandro Larena Martínez& & & \\
	\cline{2-2}
	& Mariam de la Caridad Cabrera Alarcón& & & \\
	\cline{2-2}
	& Maureen Bello Ruiz& & & \\
	\cline{2-2}
	& Dairys Cordero Tortoló& & & \\
	\cline{2-2}
	& Mariam Marrero Brito& & & \\
	\hline
	2019-2020& Ana Laura Mederos González& 9 & 61 & 15\% \\
	\cline{2-2}
	& Ann Sheyla Mirabal Navia& & & \\
	\cline{2-2}
	& Ana Laura Abreu Alfonso& & & \\
	\cline{2-2}
	& Arllete Barroso Quesada& & & \\
	\cline{2-2}
	& Arlette Ortega Arencibia& & & \\
	\cline{2-2}
	& Mariam Marrero Brito& & & \\
	\hline
	
	2021& Claudia Acebo Suárez& 5 & 62 & 9\% \\

	\cline{2-2}
	& Greter McIntosh González & & & \\

	\cline{2-2}
	& Maureen Bello Ruíz& & & \\
	\cline{2-2}
	& Gretter Hernández Cabañas& & & \\
	\cline{2-2}
	& Rocío Betancourt Rodríguez& & & \\
	\cline{2-2}
	& Nicolás de Jesús Alberto Alba& & & \\
	\hline
	2022& Melissa Laura González Díaz& 8 & 50 &  16\%\\
	\cline{2-2}
	& Ana Laura Mederos González & & & \\
	\cline{2-2}
	& Mellissa Laura González Díaz& & & \\
	\cline{2-2}
	& Fernanda Pérez Gordillo& & & \\
	\cline{2-2}
	& Mariangela Rufín Cao& & & \\
	\cline{2-2}
	& Yoanny Rodríguez Aguiar& & & \\
	\cline{2-2}
	& Gretter Hernández Cabañas& & & \\
	\cline{2-2}
	& Sheila Barrios Rodríguez& & & \\
	\cline{2-2}
	& Loraine Izquierdo Mena& & & \\
	\cline{2-2}
	& Liliana de la Caridad Orozco Cabrera & & & \\
	\hline
	\textbf{Total}& & \textbf{31}& \textbf{333} & \textbf{9.1\%} \\
	\hline
	
	\caption{Alumnos Ayudantes (Elaboración propia; Fuente: Resoluciones de alumnos ayudantes)}
	\label{tab:AlumnosAyudantes}
\end{longtable}

En la Tabla \ref{tab:AlumnosAyudantes}, en el período de 2017-2018 al 2022, se mantiene una media de formación de alumnos ayudantes en un rango entre el 14\% y el 20\%, según matrícula por cada año académico y cantidad de los mismos. De una matrícula total de 333 estudiantes en esta etapa (cursos 2017-2018 al 2022), el 9.1\% se desempeñó como alumnos ayudantes de la Educación Superior. De ellos 13 estudiantes ejercen la ayudantía por más de un curso académico, mostrando la sostenibilidad en su preparación, y después de graduados 4 se han quedado como profesores de la carrera. 

La realización de exámenes de premio ha ido en ascenso, como se evidencia en la Tabla \ref{tab:examenpremio}. Los cursos académicos más significativos son: el curso 2021, con 5 asignaturas que aplican examen de premio y con la participación de 6 estudiantes; el curso 2022, con 6 asignaturas que aplican examen de premio y con la participación de 7 estudiantes. En el curso 2021 quedan empatadas con dos 1er Premio las estudiantes Fernanda Pérez Gordillo (GSD 11) y Yamilet Naranjo Pérez (GSD 31). En el curso 2022 se destaca la estudiante Fernanda Pérez Gordillo (GSD 21) con tres 1er Premio.

\begin{longtable}{|c|c|p{5cm}|c|p{6cm}|}
	
	\endfirsthead
	
	\mc{5}{>{}c}{\tablename\ \thetable{} Continuación de la página anterior }\\ 
	
	\endhead
	
	\hline
 \underline{\textbf{Curso}}	& \underline{\textbf{Año}}  & \multicolumn{1}{|c|}{\underline{\textbf{Asignatura}}} & \underline{\textbf{Premios}} & \multicolumn{1}{|c|}{\underline{\textbf{Estudiantes}}} \\
	

		& \underline{\textbf{Académico}} & &  &  \\
	\hline
	2017-2018 & 2do año &  Pensamiento Filosófico & 1er Premio & Etianys Alfonso Dueñas\\
	\cline{2-5}
	& 2do año & Pensamiento Filosófico & 2do Premio & Dairys Cordero Tortoló\\
	\hline
	2018 - 2019 & - & - & - & - \\
	\hline
	2019 - 2020 & - & - & - & - \\
	\hline
	2021 & 1er año & Metodología de la Investigación I & 1er Premio & Fernanda Pérez Gordillo\\
	\cline{2-5}
	& 1er año & Metodología de la Investigación I & 2do Premio & Liliana de la Caridad Orozco Cabrera \\
	\cline{2-5}
	& 1er año & Introducción a la Gestión Sociocultural para el Desarrollo & 1er Premio &  Fernanda Pérez Gordillo\\
	\cline{2-5}
	& 1er año & Introducción a la Gestión Sociocultural para el Desarrollo & 2do Premio & Liliana de la Caridad Orozco Cabrera\\
	\cline{2-5}
	& 2do año & Comunicación Sociocultural & - & Melissa Laura González Díaz \\
	\cline{2-5}
	& 2do año & Comunicación Sociocultural & - & Lauren García López\\
	\cline{2-5}
	& 3er año & Gestión Medioambiental, de Salud y Prevención Social & 1er Premio & Yamilet Naranjo Pérez\\
	\cline{2-5}
	& 3er año & Gestión Medioambiental, de Salud y Prevención Social & 2do Premio & Maidelis Morera Lugo \\
	\cline{2-5}
	& 3er año & Gestión Sociocultural del Patrimonio & 1er Premio & Yamilet Naranjo Pérez \\
	\cline{2-5}
	& 3er año & Gestión Sociocultural del Patrimonio & 2do Premio & Maidelis Morera Lugo\\
	\hline
	2022 & 1er año & Metodología de la Investigación I &  1er Premio & Rachel Herrera Delgado \\
	\cline{2-5}
	& 1er año & Metodología de la Investigación I & 2do Premio & Ena María Mora Gómez\\
	\cline{2-5}
	& 1er año & Metodología de la Investigación I & 2do Premio & Helen María Santana Finalé\\
	\cline{2-5}
	& 2do año & Metodología del Trabajo Social y Comunitario & 1er Premio & Fernanda Pérez Gordillo\\
	\cline{2-5}
	& 2do año & Gestión Sociocultural del Patrimonio & 1er Premio & Fernanda Pérez Gordillo\\
	\cline{2-5}
	& 2do año & Historia de la Cultura Latinoamericana y Caribeña I & 1er Premio & Fernanda Pérez Gordillo\\
	\cline{2-5}
	& 3er año & Gestión Medioambiental, de Prevención de Salud y Social & 2do Premio &  Gretter Hernández Calaña \\
	\cline{2-5}
	& 3er año & Gestión Medioambiental, de Prevención de Salud y Social & 2do Premio & Arianna de la Caridad Fonseca Lascuncet \\
	\cline{2-5}
	& 3er año & Gestión Medioambiental, de Prevención de Salud y Social & 2do Premio & Melissa Laura González Díaz\\
	\cline{2-5}
	& 3er año & Artes Visuales Cubanas. Optativa 1 &  1er Premio & Melissa Laura González Díaz\\
	\hline
	
	\caption{Participación en exámenes de premio (Elaboración propia; Fuente: Actas de exámenes de premio en expedientes de estudiantes}
	\label{tab:examenpremio}
\end{longtable}

\begin{longtable}{|c|c|c|c|c|c|c|c|c|}
	\hline
	\underline{\textbf{Curso}}& \underline{\textbf{Matrícula}} &\mc{2}{>{}c|}{\underline{\textbf{Cantidad de}} }  & \mc{2}{>{}c|}{\underline{\textbf{Cantidad de}} }  & \underline{\textbf{Cantidad de }} & \underline{\textbf{Cantidad de }} &  \underline{\textbf{Cantidad de }}\\
	
	& &\mc{2}{>{}c|}{\underline{\textbf{trabajos}} }& \mc{2}{>{}c|}{\underline{\textbf{autores}} } & \underline{\textbf{relevantes}} & \underline{\textbf{destacados}} & \underline{\textbf{menciones}} \\
	\hline
	2022 & 50 & 19 & 38\% & 29 & 58\% & 5 & 3 & 2\\
	\hline
	\caption{Datos estadísticos de Jornadas Científicas; Fuente: Evidencias en expediente de los estudiantes}
	\label{tab:jce}
\end{longtable}

La Tabla \ref{tab:jce} muestra la participación de los estudiantes en el año base del proceso de evaluación en Jornadas Científicas Estudiantiles a nivel de carrera, de Facultad y del Departamento de Historia y Marxismo Leninismo. Contándose con una representación de un 58\% de autores con respecto a la matrícula del curso académico, representados en 19 trabajos de investigación equivalente al 38\% de dicha matrícula.

Se reconoce de igual modo en el curso 2022 la participación en eventos internacionales, nacionales y locales por parte de los estudiantes de la carrera. Entre estos eventos se destaca: 3er Foro Internacional de Experiencias Escolares en Educación para el Desarrollo Sostenible (ISIMA Universidad de México y RED Educa Verde); XIII Simposio Internacional de Educación y Cultura; I Convención Científica Internacional de la Universidad de Cienfuegos; WEBIMAR; I Taller Científico Estudiantil de las Ciencias Sociales y las Humanidades; Universidad – Sociedad (UNISOC). También se participa en concursos como: Concurso \# Arte Ciencia, de la Universidad de Pinar del Río; Concurso Fidel y los Jóvenes Universitarios, de la Universidad de Holguín.
Se han realizado actividades de prevención y consumo de drogas, se cuenta con un diagnóstico actualizado de los grupos de riesgo, y están establecidas en la estrategia educativa las acciones a realizar. Es importante destacar que el porciento de estudiantes fumadores y consumidores de alcohol por año no es significativo respecto a la matrícula, declaran que lo hacen ocasionalmente (en celebraciones, fiestas, playa, etc), o esporádicamente, cuando salen de fiesta. No obstante, sí debe ser considerado para realizar acciones de prevención, pues constituye un elemento de riesgo para todos.
Mediante la estrategia educativa se definen y precisan las acciones extensionistas dirigidas a promover una recreación saludable y otras relacionadas a la prevención. Se destaca la participación en las actividades que se realizan los 1ero de diciembre y el día Mundial de la lucha contra el SIDA, pues se incluyen acciones concretas, donde se manifiestan los modos de actuación del profesional.