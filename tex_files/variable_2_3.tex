El   trabajo científico de la carrera da respuesta a las demandas de instituciones y centros del territorio. Tiene en cuenta las prioridades establecidas en los Lineamientos de la Política Económica y Social del PCC. La investigación científica de los profesores de la carrera se consolida en los resultados obtenidos. 
El 100\% de los profesores del claustro obtienen resultados de investigación específicos que responden a las demandas externas del territorio y se encuentran asociados a proyectos tales como:

\begin{itemize}
	\setlength\itemsep{-0.5em}
	\item Proyecto Identidad y realidad cubana: estudio sociocultural del impacto de las transformaciones socioeconómicas en el centro de Cuba (PAP) (I+D+I) / M. Sc. Soilen Cedeño Solis
	\item Gestión sociocultural para el desarrollo local en el Consejo Popular Matanzas Este (PAPT) / Dr. C. Ana Gloria Peñate Villasante
	\item Estudios socioculturales para el desarrollo sostenible: UM-DECORARTE (PNAP) / Dr. C. Rosa Elvira Alfonso Ramos
	\item Patrimonio cultural y formación: patrimonio cultural universitario (PCU), historia, educación patrimonial y desarrollo local (PNAP) / Dr. C. Lissette Jiménez Sánchez
	\item VIDAS en el Rabí (PAPT) / Dr. C. Odalis Alberto Santana
	\item \emph{CCI \& CITYTOUR}:  la competencia comunicativa intercultural en el discurso de interpretación patrimonial para el desarrollo local del turismo de ciudad (PNAP) / Dr. C. Jorge Luis Rodríguez Morell
	
\end{itemize}

Estos proyectos investigativos responden a las siguientes líneas de investigación diseñadas por la Universidad: Eficiencia de los procesos tecnológicos, Perfeccionamiento del sistema educativo y Estudios sociales para el desarrollo; coincidiendo esta última con el nombre del Grupo Investigativo del departamento, cuyo objetivo es desarrollar investigaciones sobre procesos socioculturales que marcan el estado actual de nuestra sociedad. Se plantean cuatro temáticas interrelacionadas transversalmente: 

\begin{itemize}
	\setlength\itemsep{-0.5em}
	\item Desarrollo local e intervención comunitaria.
	\item Procesos culturales cubanos.
	\item Extensión universitaria.
	\item Gestión integral del patrimonio.
\end{itemize}

La actividad de posgrado mantiene un comportamiento estable en cuanto a la formación académica en la etapa que se evalúa, lo que avala el papel desempeñado por la carrera en la superación y formación posgraduada. Se destaca el incremento del porciento del claustro en la impartición de doctorados, maestrías y sus tutorías, los que tributan a las áreas del conocimiento de la carrera, al proceso de formación y a la calidad del pregrado:

\begin{itemize}
	\setlength\itemsep{-0.5em}
	\item Programa de formación Doctoral en Ciencias de la Educación (Excelencia)
	\item Programa de formación Doctoral en Ciencias Económicas (Excelencia)
	\item Maestría en Estudios Sociales y Comunitarios (Excelencia)
	\item Maestría en Educación (Excelencia)
	\item Maestría en Ciencias de la Educación Superior (Certificada)
	\item Maestría en Didáctica de las Humanidades (Excelencia)
	\item Maestría en Administración de Empresas (Excelencia)
	\item Diplomado en Gestión Sociocultural para el Desarrollo Local y Humano (Propio del departamento, direccionado a satisfacer las demandas del territorio)
	\item Diplomado Competencia comunicativa intercultural, patrimonio y turismo de ciudad
\end{itemize}

Otras áreas de superación para la carrera han sido:

\begin{itemize}
	\setlength\itemsep{-0.5em}
	\item Programa de formación doctoral en Ciencias Históricas (UH)
	\item Programa de formación doctoral en Ciencias Sociológicas (UH)
	\item Programa de formación doctoral en Ciencias de la Comunicación (UH)
	\item Maestría en Estudios Históricos y Antropología Sociocultural Cubana (UCF)
	\item Maestría en Ciencias Sociológicas (UH)
	\item Maestría en Ciencias de la Comunicación (UH)
\end{itemize}