Cada estrategia curricular se trabaja a partir de los modos y esferas de actuación profesional de un gestor sociocultural, lo que aporta conocimientos, valores y herramientas para el ejercicio de la profesión. La aplicación de las mismas se concibe con una visión integradora a partir de las potencialidades que ofrece el Plan de Estudio E, concretándose en el trabajo metodológico de los diferentes años académicos, disciplinas y asignaturas.

Las \textbf{Estrategias Curriculares de la carrera según el Documento del Plan de Estudio E} son:



\underline{\textbf{El uso de la lengua materna}}\\
El uso adecuado de la lengua materna se convierte en un recurso indispensable para poder enfrentar los requerimientos de la especialidad en sus diversos modos y esferas de actuación profesionales, en la medida que no solo garantiza lograr mayor comunicación, comprensión e interpretación en nuestro idioma tanto en sus manifestaciones orales como escritas.

Es por tanto una necesidad formativa transversal y permanente de la carrera atendiendo a la calidad que requiere un profesional de las Ciencias Sociales. La estrategia se aplica en todas las asignaturas de la carrera, potenciando el uso del lenguaje técnico propio del perfil profesional, en las evaluaciones realizadas, lo que hace posible el desarrollo y la interacción del pensamiento, la comunicación, la comprensión y la expresión de los profesionales de la gestión sociocultural en su actuar cotidiano

\underline{\textbf{El uso de la tecnología de la información}}\\
El uso adecuado y pertinente de la tecnología de la información es una exigencia general para todo egresado universitario, lo que se refleja para esta carrera en la concepción de una disciplina de computación en el currículo base, pero también desde asignaturas como Gestión de la Información y el Conocimiento, Metodología de la Investigación y con la optativa Estadística Aplicada a las Ciencias Sociales. Con esta estrategia se alcanzan resultados en la formación profesional de los estudiantes, concretándose en resultados como participación en eventos, uso del correo electrónico, navegación y búsqueda de información en sitios referenciados, trabajo en la plataforma interactiva Moodle, entre otros. Se ha logrado que los estudiantes puedan trabajar con bases de datos remotas y locales, bibliotecas personales digitalizadas, así como el uso de materiales y multimedias. Se evidencia un adecuado nivel de satisfacción de los estudiantes con la orientación que se les ofrece en cada una de las disciplinas y asignaturas para el trabajo con los recursos informáticos. El uso de las tecnologías de la información se considera como aspecto de evaluación sistemático en todas las disciplinas y asignaturas de la carrera.

\underline{\textbf{Estrategia de idioma inglés}}\\
Como resultado del perfeccionamiento de los planes de estudio en la Educación Superior de Cuba, que tiene por objetivo alcanzar la formación integral de los futuros profesionales, se tiene en cuenta la necesidad de que los egresados muestren competencia comunicativa en una lengua extranjera, principalmente en idioma inglés por ser la de más amplia difusión internacional a la luz de las necesidades y proyecciones del desarrollo del país y en consonancia con las tendencias internacionales el dominio de esta lengua se convierte en un imperativo de primer orden. 

En el diseño de los Planes de Estudio E se concibe que la disciplina Idioma Inglés no tenga presencia en el currículo y se considere como requisito de graduación el dominio de este idioma en un nivel pre establecido. Por tanto, esta estrategia se trabaja de manera individualizada, a partir de las necesidades concretas de cada estudiante, a partir de los diagnósticos realizados por los profesores del Centro de Idiomas, que determinan el nivel de cada estudiante y las acciones a seguir.
 
Igualmente, desde la carrera se potencia el trabajo con esta estrategia desde cada disciplina y cada una de las asignaturas, con frases para interpretar, búsquedas bibliográficas especializadas en temáticas específicas, en cada trabajo de curso y diploma se entrega un resumen en idioma inglés, lo que permite ampliar el espectro cultural y profesional del futuro gestor sociocultural, dado la variedad de su perfil profesional.

\underline{\textbf{Estrategia para la educación ambiental}}\\
La educación ambiental se convierte en recurso profesional de la carrera y por ello exige una adecuada y sostenida formación desde las disciplinas y en el desarrollo de la actividad laboral. Desde actividades extensionistas y académicas se promueve la educación ambiental en todos los años de la carrera.

En el currículo se imparten asignaturas como Gestión Medioambiental, de Salud y Prevención Social, Naturaleza y Sociedad, Patrimonio Arqueológico Matancero, Políticas Sociales y Públicas, Estudios Poblacionales, Metodología del Trabajo Social y Comunitario, Historia y Cultura Regional, Gestión Sociocultural del Patrimonio, que direccionan el trabajo con esta estrategia de manera particular. Hay presentaciones en jornadas científicas estudiantiles, actividades concretas orientadas desde la práctica laboral investigativa y trabajos de diplomas de estudiantes con resultados específicos referidos a este tema que forman parte del Observatorio Ambiental COSTATENAS.

Los estudiantes de la carrera desarrollan acciones en sus Consejos Populares para potenciar la educación ambiental y la prevención de salud entre sus habitantes, donde existe círculos de interés con niños para trabajar desde edades tempranas estos temas

Además de las Estrategias Curriculares definidas en el Documento del Plan de Estudio E, la carrera trabaja las siguientes estrategias:

\underline{\textbf{Estrategia de Historia de Cuba}}\\
Se cumplimenta durante todo el desarrollo del programa al Incorporar las nociones y elementos de los sistemas de conocimientos históricos a los contenidos de la asignatura Historia de Cuba según corresponden , al vincular la relación de la situación socio-histórica con el desarrollo de la Política Cultural Cubana y la evolución de los procesos culturales que desemboca en la concepción de potenciar el trabajo en las comunidades mediante los proyectos socioculturales en busca de la transformación individual y social. Utilizar los elementos de la historia local vinculados a las comunidades donde se realiza diagnóstico, y existen varios proyectos socioculturales de relevancia, además de hacer alusión a las fechas históricas, según los días de clases. 

\underline{\textbf{Estrategia de Formación Jurídica}}\\
Se trabaja con intencionalidad en el marco legislativo para la gestión de proyectos en Cuba. Intensificando en el estudio de la Ley No. 44/2012 del CITMA. Se identifican las fuentes de financiamiento para el desarrollo de proyectos culturales y se explica la diversidad de la cooperación internacional para el financiamiento de proyectos socioculturales.

\underline{\textbf{Estrategia de Preparación para la Defensa}}\\
Al propiciar el estudio e investigación de las comunidades y la creación de proyectos socioculturales comunitarios, como una forma de facilitar procesos de transformación individual y social, se propicia el afianzamiento del sentido de pertenencia, el amor al barrio, a los valores patrios y se trabaja por el respeto a las personalidades y la defensa de la identidad y cultura nacionales. También como parte de la estrategia de defensa se trabaja en la prevención de las indisciplinas sociales, adicciones, enfermedades degenerativas y la divulgación de estilos de vida saludables. De esta forma se trabaja desde la asignatura por la continuidad del proyecto social cubano.

\underline{\textbf{Estrategia de Formación Económica y de Dirección}}\\
Se trabaja en la potenciación de las habilidades a lograr con las asignaturas de la Disciplina Principal Integradora, en Gestión Sociocultural, Gestión de Proyectos y Evaluación de Impactos y Gestión Organizacional y de Gobierno de manera particular, aunque en todas las del currículo se contribuye y potencia el trabajo con esta estrategia. La gestión de proyectos socioculturales comunitarios tiene como principio metodológico fundamental el análisis de los procesos socioculturales en su relación y determinación, en última instancia por los procesos económicos. Además, al aplicar esta concepción se profundiza en el conocimiento de los procesos de dirección, en sus diferentes funciones básicas: planeación, organización, dirección y control.\\

\textbf{Fortalezas:}
\begin{itemize}
	\setlength\itemsep{-0.5em}
	\item El PPD garantiza la formación integral de los estudiantes y la apropiación de las esferas y modos de actuación profesional. Se garantiza igualmente la superación posgraduada, con un diseño planificado y coherente que responde al perfil profesional y necesidades del territorio.  
	\item Elevado predominio y liderazgo de profesores con categoría docente Titular y Auxiliar en la coordinación del trabajo metodológico, a nivel de departamento, carrera, colectivos de año y disciplinas.
	\item La estrategia educativa se concibe como un sistema que rige el trabajo metodológico, con un adecuado balance entre sus dimensiones, para cumplir con calidad los objetivos de la formación del profesional y el objetivo integrador de cada año académico
	\item Alto prestigio de las unidades docentes de la carrera, que contribuyen significativamente a la formación de los estudiantes en sus modos y esferas de actuación. En el claustro existen profesores a tiempo parcial que son profesionales que laboran en estas UD y que son fundadores de la carrera, lo que garantiza una excelente relación entre la carrera y su impacto en el territorio matancero, avalado con resultados concretos de investigación. 
	\item Excelente labor de la DPI en la formación curricular e investigativa de los estudiantes, desde la práctica laboral investigativa hasta el ejercicio de culminación de estudios con sus diferentes tipologías (Trabajo de Diploma, Examen Estatal y Portafolio). Para el logro efectivo de la culminación de estudios se tiene en cuenta la ubicación laboral anticipada de cada estudiante y cómo da respuesta a las problemáticas de la institución en particular y del territorio en general.
	
	Tiene gran valía para la carrera la realización cada año de las defensas públicas del plan de estudios, con empleadores, claustro y estudiantes, lo que permite un acercamiento constante y actualización a los intereses del territorio, necesidades de los empleadores y expectativas de los estudiantes. A partir de ellas se perfecciona el currículo propio y optativo / electivo, el cual se gestiona de forma flexible y participativa, respondiendo a los objetivos de la formación del profesional y a las necesidades del territorio.
\end{itemize}


\textbf{Debilidades:}
\begin{itemize}
	\setlength\itemsep{-0.5em}
	\item No se declaran 
\end{itemize}