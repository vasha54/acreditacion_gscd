La estrategia educativa de la carrera tiene como objetivo fortalecer la labor educativa y la formación de una cultura general integral de los estudiantes de la carrera de Gestión Sociocultural para el Desarrollo. La misma se fundamenta en la formación política ideológica con un enfoque integral en el proceso formativo en sus tres dimensiones (curricular, político-ideológica y extensionista) y con acciones definidas en cada una de ellas. Se realiza sobre la base de la unidad entre la educación y la instrucción, teniendo como instrumento básico la orientación elaborada en la Universidad de Matanzas, donde se disponen los criterios para el diagnóstico que se realiza en cada año académico. 

En consecuencia, de la estrategia educativa de la carrera se confeccionan las de los diferentes años académicos, a partir de los criterios dispuestos para este diagnóstico inicial que refleja las necesidades e intereses de los estudiantes y en correspondencia con el objetivo integrador del año. Son elaboradas por los estudiantes conducidos por los profesores guías (PG) y profesores principales de año (PPA). Estas estrategias educativas se ajustan a las necesidades metodológicas del colectivo de año y a las particularidades de los estudiantes, con un adecuado balance entre los componentes que responden a las necesidades e intereses de los estudiantes y los profesores.
 
En las estrategias educativas se logra que los estudiantes cumplan con calidad los objetivos trazados en el año, garantizando la calidad del proceso docente educativo, con acciones concretas, generales e individualizadas, en cada una de las dimensiones dispuestas para ello (curricular, extensionista y sociopolítica). Igualmente se cuentan con criterios de medidas que sirven de base para la evaluación integral de los estudiantes que se realiza al final del curso académico, lo que define el complimiento de la estrategia. 

Los estudiantes son los principales protagonistas y artífices fundamentales de las estrategias educativas, lo que garantiza la efectividad de las mismas y que se parezcan a cada grupo. Se priorizan en ella las tareas de impacto, la vinculación a líneas y proyectos de investigación y las orientaciones de la práctica laboral investigativa en correspondencia con el objetivo integrador del año.