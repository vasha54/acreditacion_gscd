La experiencia y estabilidad del claustro satisface las exigencias del proceso de formación profesional. Está avalada por las categorías docentes, científicas y especialización en las áreas del conocimiento sociocultural, psicológico, sociológico, histórico y patrimonial. Los miembros del claustro han desempeñado y desempeñan responsabilidades a nivel de departamento, facultad, universidad y en las organizaciones políticas y de masa.

Se destacan profesores a tiempo parcial por su grado científico, experiencia profesional y reconocimiento social, entre ellos, fundadores de la carrera y procedentes de la enseñanza superior.

Los colectivos pedagógicos de año y las disciplinas son dirigidos por los profesores de más experiencia y mayor categoría docente, lo cual satisface las exigencias del proceso de formación del profesional. La Disciplina Principal Integradora de la carrera está constituida por profesores con dominio de los objetivos, contenidos y la organización propia de cada año académico. 

\begin{longtable}{|c|c|p{7cm}|c|}
	
		\endfirsthead
	
	\mc{4}{>{}c}{\tablename\ \thetable{} Continuación de la página anterior }\\ 
	\hline
	\endhead
	
	\hline
	\underline{\textbf{Año}} &  \underline{\textbf{Período}}& \multicolumn{1}{|c|}{\underline{\textbf{Nombres y Apellidos}}} & \underline{\textbf{Categoría Docente}} \\ \hline
	\mc{4}{|>{}r|}{Continúa en la siguiente página }\\  
	\hline
	Primero &1ero & Dr. C. Rosa Elvira Alfonso Ramos& Titular \\ \cline{2-3}
	& 2do& Dr. C. Rosa Elvira Alfonso Ramos& \\ \hline
	Segundo& 1ero& Dr. C. Walfredo Gonzáles Hernández& Titular \\ \cline{2-4}
	& 2do& M. Sc. Yoanis Almeida Roldán& Auxiliar\\ \hline
	Tercero& 1ero& M. Sc. Madelín Rodríguez Benítez& Auxiliar\\ \cline{2-4}
	& 2do& Lic. Guillermo Alfredo Jiménez Pérez& Asistente\\ \hline
	Cuarto&1ero & Dr. C. Evelyn González Paris& Titular\\ \cline{2-4}
	& 2do& Dr. C. Ana Gloria Peñate Villasante& Titular\\ \hline

	\caption{Claustro de la Disciplina Principal Integradora (Elaboración propia)}
\end{longtable}


La experiencia profesional del claustro, está avalada por pertenecer: 

\begin{itemize}
	\item Miembro de la Cátedra de Estudios Etnoantropológicos \emph{Fernando Ortiz}, Universidad de Matanzas.
	\item Miembro del Grupo de Trabajo del Proyecto de la Ruta del Esclavo en la provincia de Matanzas.
	\item Miembro del Comité Científico del Fórum Fernando Ortiz: Cultura Popular Tradicional e Historia Nacional.
	\item Miembro de Comité Organizador y Científico en numerosos eventos científicos nacionales e internacionales: Taller Internacional de las Disciplinas Humanísticas, UM; Red Iberoamericana de Pedagogía (REDIPE).
	\item Potencial Científico de la Dirección Provincial de Cultura. Matanzas.
	\item Miembro de la Comisión de postgrado de la UM.
	\item Miembro de la Comisión de Grados Científicos de la UM.
	\item Miembro del Consejo Científico de Ciencias Sociales.
	\item Presidente y miembro del Consejo Científico de la Universidad de Matanzas
	\item Miembro de la Asociación de Lingüistas de Cuba.
	\item Miembro de la Asociación de Pedagogos de Cuba.
	\item Miembro de la Sociedad Cultural José Martí.
	\item Miembro de la Unión de Escritores y Artistas de Cuba (UNEAC).
	\item Miembro de la Asociación de Comunicadores de Cuba (ACC).
	\item Experto de la Junta de Acreditación Nacional.
	\item Coordinador del Diplomado en Gestión Sociocultural para el Desarrollo Local y Humano.
	\item Coordinador de la Maestría Didáctica de las Humanidades.
	\item Presidente, Miembro y Oponente en tribunales de defensa de Maestría en Ciencias de la Educación Superior.
	\item Miembro permanente del tribunal de Doctorado en Ciencias de la Educación y Ciencias Económicas.
\end{itemize}