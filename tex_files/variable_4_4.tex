La Universidad de Matanzas cuenta con la residencia estudiantil en edificios de reciente restauración, que presentan las condiciones necesarias para la estancia de los estudiantes. El campus universitario presta servicios médicos y estomatológicos, las áreas generales cumplen con los parámetros de seguridad y movilidad requerida. El personal tiene acceso a un comedor y existen distintas cafeterías, tanto estatal como por cuenta propia, que se suman a las ofertas de alimentos.

El Departamento de Estudios Socioculturales se encuentra en el Edificio A, en la Facultad de Ciencias Sociales y Humanidades, donde se distribuye el personal entre el local de profesores y el de los Jefes de Departamento, además del área de dirección de la facultad, donde se encuentra el Decanato, Vicedecanato Docente y Vicedecanato de Investigación y la Secretaría Docente. Los locales están bien iluminados, instalación eléctrica reciente, cuentan con buena ventilación, estética e higiene.

Contamos con otras instalaciones que brindan otros centros vinculados a las unidades docentes, como las salas de consulta de materiales del Archivo Histórico de Matanzas, las aulas del Centro de Superación para la Cultura, las aulas del Museo Palacio de Junco, EMPAI, Museo de Bomberos, Casa Social de la UNHIC, salones de la delegación territorial del CITMA, todas equipadas con los materiales necesarios para el trabajo de los estudiantes y el personal docente.
 
Contamos en la universidad con el Parque Científico Tecnológico de Matanzas, en el cual se ofrecen diversos servicios y funcionan diversos proyectos en los que se involucran estudiantes y profesores de la carrera que forman parte de la bolsa de especialistas, entre ellos:

\begin{itemize}
	\setlength\itemsep{-0.5em}
	\item SAPGAE: realiza actividades de gestión sobre las entidades en cumplimiento del cuidado medioambiental 
	\item Proyecto Bienestar: sobre la informatización de los trámites y servicios estatales, mediante el uso de plataformas que potencien el gobierno electrónico.
	\item Creación del Observatorio Ambiental Provincial: en desarrollo actualmente y vinculado con el Observatorio Ambiental COSTATENAS, dedicado a la investigación e intercambio científico sobre las zonas costeras.
\end{itemize}

Estas instalaciones solo se encuentran disponibles en la Universidad de Ciencias Informáticas (UCI) en La Habana y en la Universidad de Matanzas, constituyendo una fortaleza importante.\\

\textbf{Fortalezas:}

\begin{itemize}
	\setlength\itemsep{-0.5em}
	\item Elevada calidad del proceso de formación del profesional a partir de la infraestructura existente en la universidad y unidades docentes. Se asegura por diferentes vías la bibliografía de las asignaturas de la carrera, con disponibilidad de acceso a la Intranet-UM e Internet. También se utiliza como bibliografía la producción científica del claustro de profesores. Existe diversidad de materiales docentes de calidad en plataformas digitales y las asignaturas del currículo están montadas en su mayoría en la Moodle, con un 100\% de interactividad de los estudiantes, lo que garantiza el funcionamiento del proceso docente educativo. 
	\item Se cuenta con una infraestructura en el espacio universitario de reciente restauración, capaz de brindar los recursos necesarios para la formación de los profesionales y la investigación, tales como: aulas propias, laboratorios, residencia estudiantil, instalaciones médicas, biblioteca con material bibliográfico acorde a las disciplinas que trabajan los profesionales de la gestión sociocultural, plataformas digitales interactivas que potencian la formación y que brindan alternativas para el acceso a materiales, complementado por los locales que ofrecen las unidades docentes, dirigidos por profesionales reconocidos en el ámbito investigativo y cultural. 
	\item Es elevado el número de zonas WiFi con que cuenta la universidad, garantizando la posibilidad de conectividad de estudiantes y profesores.
	\item Elevado incremento de la cuota de navegación y continuo perfeccionamiento de los servicios informáticos, unido a un canal de navegación con un total de 100Mb/s, garantiza la velocidad de navegación y descarga apropiada para obtener distintos materiales y recursos educativos, con amplia disponibilidad de los locales para el uso del personal universitario.
	\item Importante vínculo con instituciones en el extranjero como resultado de las becas de investigación (Universidad de Valencia). Esto permite el acceso a sus bibliotecas virtuales, posibilitando la descarga de bibliografía actualizada tanto para el proceso docente como la superación profesional del claustro.
\end{itemize}

\textbf{Debilidades:}
\begin{itemize}
	\setlength\itemsep{-0.5em}
	\item No se declaran
\end{itemize}