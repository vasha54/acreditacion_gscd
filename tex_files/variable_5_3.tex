La carrera gestiona su proceso docente - educativo con una evidente y efectiva relación entre los componentes académico, laboral e investigativo, que responden a la flexibilidad, calidad y pertinencia exigida en la formación del profesional que demanda el territorio. Los colectivos pedagógicos desarrollan su trabajo docente - metodológico dirigido a su constante perfeccionamiento, destacándose el papel de la Disciplina Principal Integradora como columna vertebral en la formación de la habilidad profesional, la educación a través de la instrucción, la formación de valores, los métodos de enseñanza, las formas organizativas de la enseñanza, los sistemas de evaluación del aprendizaje, el uso de las tecnologías de la información y las comunicaciones y la aplicación de las estrategias curriculares así como el trabajo con los documentos rectores del Proyecto Social Cubano.