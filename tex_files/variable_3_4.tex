La carrera lleva a cabo el trabajo metodológico en este nivel, con el objetivo de lograr el cumplimiento con calidad del Modelo del Profesional, dirigiendo así el trabajo de las disciplinas y los años.

El trabajo metodológico de las disciplinas tiene el propósito de cumplir con calidad los objetivos generales de la misma y trabaja para lograr coherencia en la integración y sistematización de contenidos de las diferentes asignaturas en el proceso de formación.

Los colectivos de años propician unificación de los aspectos educativos e instructivos con un enfoque multidisciplinario., analizando el proceso docente educativo y la situación de cada estudiante. De esta manera valoran y toman las medidas para el mejoramiento continuo de la calidad de dicho proceso y el resultado de su evaluación integrada, que se realiza a partir del cumplimiento de la estrategia educativa del año.

La evaluación del progreso de los estudiantes en cada una de las asignaturas del año, aunque básicamente parte de los resultados docentes alcanzados por ellos, presta atención a todos los escenarios educativos: la residencia estudiantil, la actividad laboral investigativa, tareas de impacto social, extensión universitaria y en las relaciones socio-familiares.

De esta manera se contribuye a la formación de una cultura general integral de los estudiantes, con atención diferenciada y propiciando el trabajo en equipo.\\

\textbf{Fortalezas}

\begin{itemize}
	\item Los estudiantes participan en la solución de los problemas del territorio, vinculados a su profesión. (Destacándose la participación en INNOVAMATANZAS (2018) aplicando encuestas a la población y presentándose los resultados de esta tarea en el taller de cierre del evento. En el curso 2019-2020, se encuestaron a jóvenes matanceros para un estudio Nacional sobre adolescencias y juventudes, coordinado por la UJC. En el curso 2022, los estudiantes de 4to año participaron en las actividades realizadas en la provincia de Matanzas por el Movimiento Sembrar Con Ciencia. Se destacaron en las labores en apoyo a la COVID 19, incorporados a las pesquisas, a la zona roja, al trabajo en organopónico y a otras actividades convocadas para el enfrentamiento a la pandemia. Además de acciones de apoyo comunitario integrado realizadas durante el curso 2021).
	
	\item Aumenta la cantidad de asignaturas que realizan exámenes de premio con respecto al período de evaluación anterior. (En los cinco cursos evaluados hasta el 2015-2016, solo las asignaturas Historia Universal de América, Inglés I y Filosofía y Sociedad lo aplicaron. Sin embargo, en esta etapa que se evalúa,  en el curso 2022 llegan a ser seis asignaturas: Metodología de la Investigación Social I, Gestión Sociocultural del Patrimonio, Historia de la Cultura Latinoamericana y Caribeña I, Gestión Medioambiental de Prevención de Salud y Social y Artes Visuales Cubanas). 
	
	\item Aumenta la cantidad de alumnos ayudantes con respecto a los cinco años evaluados hasta el curso 2015-2016, solo ejercían ayudantía 9 estudiantes. Actualmente, se reconoce por estudiantes y tutores la calidad del desempeño de los alumnos ayudantes representados por 31 estudiantes en dicho movimiento en la Educación Superior (De ellos 13 estudiantes ejercen la ayudantía por más de un curso académico, mostrando la sostenibilidad en su preparación. Se han quedado después de graduados 4 como profesores de la carrera). En la Enseñanza Media 20 estudiantes integran el contingente Educando por Amor, en el quinquenio que se evalúa.
	
	\item Aumenta la participación de los estudiantes en el movimiento de artistas aficionados, representando para el curso 2022 el 20\% de la matrícula del curso con participación activa en más de una manifestación artística (artes visuales, música, literatura y audiovisual), en el festival a nivel de Facultad y concursando a nivel de Universidad.
	
	\item El 100\% de los estudiantes aprueban el ejercicio de certificación de idioma inglés, los ejercicios integradores y culminación de estudios, en los dos últimos prevalecen los resultados de 4 y 5 puntos. 
	
	\item El 100\% del trabajo científico-investigativo y profesional que realizan los estudiantes responde a las problemáticas principales de las instituciones del territorio y fortalece la pertinencia social de la carrera, relacionándose con los modos y esferas de actuación del profesional. Demostrado en que el 100\% de los estudiantes en el curso 2022 se encuentran incorporados a proyectos de investigación donde desarrollan pesquisas sobre temas de desarrollo local e intervención comunitaria, procesos culturales cubanos, extensión universitaria, gestión integral del patrimonio; resultados científicos que se exponen en eventos y se presentan como temas de trabajos de diploma. Forman parte de las sociedades científicas estudiantiles.
	
	\item El 100\% de los estudiantes que se evalúan muestran el desarrollo de habilidades profesionales y de los modos y esferas de actuación profesional en las prácticas laborales investigativas (solo son no evaluados los que están de licencia de matrícula o no presentados a exámenes que posteriormente no continúan la carrera. Por ejemplo, en el 2do período del curso 2022, en 1er año, el 100\% de los estudiantes es evaluado; en 2do año 4 estudiantes no se presentan y son baja para el inicio del curso 2023; en 3er año, 1 estudiante no se presenta y ocurre lo mismo que en los casos anteriores).
\end{itemize}

\textbf{Debilidades}

\begin{itemize}
	\item No se declaran.
\end{itemize}