La carrera lleva a cabo el trabajo metodológico en este nivel, con el objetivo de lograr el cumplimiento con calidad del Modelo del Profesional, dirigiendo así el trabajo de las disciplinas y los años.
El trabajo metodológico de las disciplinas tiene el propósito de cumplir con calidad los objetivos generales de la misma y trabaja para lograr coherencia en la integración y sistematización de contenidos de las diferentes asignaturas en el proceso de formación.

Los colectivos de años propician unificación de los aspectos educativos e instructivos con un enfoque multidisciplinario., analizando el proceso docente educativo y la situación de cada estudiante. De esta manera valoran y toman las medidas para el mejoramiento continuo de la calidad de dicho proceso y el resultado de su evaluación integrada, que se realiza a partir del cumplimiento de la estrategia educativa del año.

La evaluación del progreso de los estudiantes en cada una de las asignaturas del año, aunque básicamente parte de los resultados docentes alcanzados por ellos, presta atención a todos los escenarios educativos: la residencia estudiantil, la actividad laboral investigativa, tareas de impacto social, extensión universitaria y en las relaciones socio-familiares.

De esta manera se contribuye a la formación de una cultura general integral de los estudiantes, con atención diferenciada y propiciando el trabajo en equipo.\\

\textbf{Fortalezas}

\begin{itemize}
	\setlength\itemsep{-0.5em}
	\item Destacada participación de los estudiantes en la solución de los problemas del territorio vinculados a su profesión, tales como: aplicar encuestas a la población para estudios de interés territorial y nacional, en las actividades del Movimiento Sembrar Conciencia, en las labores para el enfrentamiento a la pandemia Covid-19 y en acciones de apoyo comunitario integrado.
	\item Se consolida el sistema de alumnos ayudantes, (13 de ellos ejercen la ayudantía por más de un curso académico y 4 se han quedado después de graduados como profesores de la carrera) (mejora respecto a evaluación anterior). 
	\item El 100\% de los estudiantes aprueban el ejercicio de certificación de idioma inglés, los ejercicios integradores y de culminación de estudios, (mejora respecto a evaluación anterior).  
	\item  El 100\% de los estudiantes en el curso 2022 se encuentran incorporados a proyectos de investigación y a sociedades científicas, (mejora respecto a evaluación anterior).
	\item Alta retención y aceptable eficiencia vertical.
\end{itemize}

\textbf{Debilidades}

\begin{itemize}
	\setlength\itemsep{-0.5em}
	\item La eficiencia académica y de ciclo tiende a disminuir..
\end{itemize}