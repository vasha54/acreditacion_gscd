La carrera Gestión Sociocultural para el Desarrollo en la Universidad de Matanzas, se presenta en correspondencia con las características del eslabón de base de la profesión, como un programa de formación de pregrado dirigido a preparar un profesional comprometido socialmente y capaz de atender e incidir sobre los aspectos socioculturales presentes en los proyectos, acciones y procesos dirigidos al desarrollo social, principalmente a escala local y comunitaria. Para ello utiliza las herramientas profesionales y modos de hacer procedentes de diversas ciencias sociales.
El diseño curricular se estructura a partir de lo concebido por la Comisión Nacional de Carrera y teniendo en cuenta los documentos rectores de trabajo del Ministerio de Educación Superior. Para ello el colectivo de carrera ha realizado una gestión curricular particularizada y atendiendo a las necesidades del territorio, contando con el diseño de un currículo base, propio y optativo / electivo. 

\section{Gestión curricular en la carrera y en el colectivo pedagógico}

La estructura curricular de la carrera se encuentra en coherencia con lo concebido en el Modelo del Profesional y en correspondencia con las demandas del territorio, dirigidos en su esencia a potenciar el desarrollo humano, incidiendo en el enriquecimiento espiritual, fortalecimiento de la identidad cultural y nacional, la calidad de la vida colectiva y capacidad de participación de la población en el desarrollo social en sentido general.

El currículo asegura el trabajo con los modos y esferas de actuación, con un carácter sistémico, integrador, flexible y en función de la unidad entre la educación y la instrucción. En el currículo base se incluyen disciplinas de carácter específico (Gestión Sociocultural, Metodología Social, Desarrollo y Políticas Sociales e Historia Cultural y Pensamiento Social) y de carácter general con contenido filosófico, histórico y otras con aportes necesarios de las ciencias en particular y que posibilitan contenidos indispensables para desempeñarse como gestores socioculturales. El currículo propio está diseñado en función de las necesidades y potencialidades del territorio, teniendo en cuenta para ello las características del claustro. En este particular se cuenta con cinco asignaturas propias. El currículo optativo / electivo está pensado para el desarrollo de una cultura general integral en los estudiantes, teniendo en cuenta complementa la formación profesional. Se cuenta con el diseño de cinco asignaturas entre optativas y electivas. De manera general, el diseño curricular permitirá el actuar de forma progresiva en la medida que se logra una formación continua, respondiendo a la realidad sociocultural de los espacios donde este gestor incida. 

Los objetivos del Modelo del Profesional se concretan en las disciplinas, colectivos de años y carrera, en correspondencia con los modos y esferas de actuación, con una labor formativa consecuente con el perfil profesional y las estrategias curriculares. Los objetivos y contenidos de las disciplinas y las asignaturas dan respuesta a los requerimientos de la carrera y están en correspondencia con el objetivo integrador de cada año académico. La Disciplina Principal Integradora (Gestión Sociocultural) tiene como punto de referencia el objeto de trabajo del egresado, pues integra, controla y responde por el dominio y desarrollo de los modos y esferas de actuación de la profesión. Para ello cuenta con la presencia de asignaturas en todos los años académicos y con una estrategia orientada al cumplimiento y desarrollo de la práctica laboral investigativa. Su propósito principal es la formación de la habilidad rectora de la carrera (la gestión sociocultural), que su formación y desarrollo transversaliza la carrera con un vínculo permanente entre la teoría y la práctica.

Los colectivos pedagógicos en todos los niveles están dirigidos por profesores con elevada experiencia en la función que desempañan y en su 100\% con categoría docente superior. En la tabla se refleja la composición de los colectivos pedagógicos de carrera, año y disciplinas en correspondencia con las exigencias y estándares de calidad establecidos por el SEA-CU:



\begin{longtable}{|c|p{4cm}|c|p{6cm}|}
	
		\endfirsthead
	
	\mc{4}{>{}c}{\tablename\ \thetable{} Continuación de la página anterior }\\ 
	
	\endhead
	
	
	
	\hline
\underline{\textbf{Colectivo Carrera}}	& \mc{1}{>{}c|}{\underline{\textbf{Profesor}}} &  \underline{\textbf{Categoría Docente}} & \underline{\textbf{Categoría Científica}}  \\
	\cline{2-4}
	& Yinela Castillo Lozano & Profesora Auxiliar & Máster en Estudios Históricos y Antropología Sociocultural Cubana
	Doctaranda en Programa Doctoral de Historia  \\
	
	
	\hline
	
	
	\underline{\textbf{Colectivo de Año}} &  \mc{1}{>{}c|}{\underline{\textbf{Profesor}}} & \underline{\textbf{Categoría Docente}} & \underline{\textbf{Categoría Científica}}  \\
		\hline 
	
	1er Año & Yailet Morales Delgado & Profesora Auxiliar  & Máster en Educación Superior
	Doctaranda en Programa Doctoral Ciencias de la Educación  \\
	\hline
	2do Año &Gerardo Antonio Mier Daubar & \parbox[t]{3.5cm}{Profesor Asistente (Con 43 años de experiencia en educación y 20 años de experiencia específicamente en la Educación Superior)} & Máster en Ciencias de la Educación Superior \\
	\hline
	3er Año & Yara Antonia Alfonso Cobas & Profesora Auxiliar & Máster en Desarrollo Comunitario  \\
	\hline
	4to Año & Ana Gloria Peñate Villasante  & Profesora Titular  & Doctora en Ciencias de la Educación \\
	\hline
	\underline{\textbf{Colectivo Disciplina}} &  \mc{1}{>{}c|}{\underline{\textbf{Profesor}}}  & \underline{\textbf{Categoría Docente}}  & \underline{\textbf{Categoría Científica}}  \\
	\hline
	Gestión Sociocultural & Yailet Morales Delgado & Profesora Auxiliar & Máster en Educación Superior
	Doctaranda en Programa Doctoral Ciencias de la Educación \\
	\hline
	Metodología Social & Odalis Alberto Santana & Profesor Titular & Doctora en Ciencias de la Educación \\
	\hline
	\parbox[t]{3cm}{Desarrollo y Políticas Sociales} & Yara Antonia Alfonso Cobas  & Profesora Auxiliar  & Máster en Desarrollo Comunitario  \\
	\hline
	\parbox[t]{3cm}{Historia Cultural y Pensamiento Social} & Silvia Teresita Hernández Godoy & Profesora Titular & Doctora en Ciencias Históricas \\
		\hline 
	
	\parbox[t]{3cm}{Marxismo - Leninismo}& María Felicia Ibáñez Matienzo & Profesora Auxiliar & Máster en Desarrollo Comunitario \\
	\hline
	Historia de Cuba & Oscar Andrés Piñeira Hernández & Profesor Titular & Doctor en Ciencias Históricas \\
	\hline
	Computación & Lázaro Tió Torriente  & Profesor Titular & Doctor en Ciencias de la Educación \\
	\hline
	\parbox[t]{3cm}{Estudios de la Lengua Española } & Rosa Elvira Alfonso Ramos & Profesora Titular & Doctora en Ciencias Pedagógicas  \\
	\hline
	\parbox[t]{3cm}{Preparación para la Defensa} & Luis Orlando Milián Zambrana & Profesor Asistente & Máster en Estudios Sociales y Comunitarios \\
	\hline
	Educación Física & Ángel Fidel Llanos González & Profesor Asistente & Máster en Ciencias de la Educación Superior \\
	\hline
	\caption{Colectivos pedagógicos de la carrera en la Universidad de Matanzas (Elaboración propia)} 
	\label{tableclaustro}
\end{longtable}


El trabajo metodológico de la carrera en los diferentes niveles organizativos se articula con las prioridades de la Universidad y de la Facultad. Es un proceso sistemático que potencia las relaciones inter, intra y transdiciplinarias en torno a la Disciplina Principal Integradora y el desarrollo de la habilidad profesional integradora, lo que se expresa en el nivel de actualización constante de las asignaturas que conforman el currículo base, propio y optativo / electivo de la carrera, en torno al cumplimiento del objetivo integrador del año académico y en relación con los trabajos de curso y la práctica laboral investigativa. El trabajo metodológico sistemático favorece el proceso docente educativo en relación a:

\begin{itemize}
	\setlength\itemsep{-0.5em}
	\item Las investigaciones pedagógicas realizadas en las disciplinas han estado enfocadas al perfeccionamiento de las asignaturas, la interdisciplinariedad, desarrollo de valores, elaboración de materiales didácticos, diseños de programas, a partir de la experiencia acumulada.
	\item Se encuentran correctamente estructurados los programas de asignaturas en función del trabajo metodológico que se realiza a nivel de las disciplinas y carrera.
	\item En los programas de asignaturas aparece declarado el sistema de conocimientos, habilidades y valores a potenciar en los estudiantes, además de las estrategias curriculares y la forma en que se implementan.
	\item Las evaluaciones frecuentes, parciales y finales garantizan la medición del cumplimiento de los objetivos declarados por asignaturas y disciplinas en relación con el objetivo integrador del año académico.
	\item En cada una de las clases y actividades se emplean de forma efectiva los métodos, formas organizativas, medios y sistemas de evaluación.
	\item Se exige la utilización de las TICs en las asignaturas, trabajos de curso o diploma.
	\item Se realiza un control de la organización y proyección del colectivo para el cumplimiento con calidad del objetivo integrador del año y la integración de los aspectos educativos e instructivos con un enfoque interdisciplinario.
	\item El ejercicio de culminación de estudios se realiza en correspondencia con las diferentes líneas de investigación de la carrera y de las instituciones, organismos, empresas y organizaciones del territorio, teniendo en cuenta para ello la ubicación laboral anticipada y la participación en proyectos de investigación.
\end{itemize}

\section{Estrategia educativa de la carrera}

La estrategia educativa de la carrera tiene como objetivo fortalecer la labor educativa y la formación de una cultura general integral de los estudiantes de la carrera de Gestión Sociocultural para el Desarrollo. La misma se fundamenta en la formación política ideológica con un enfoque integral en el proceso formativo en sus tres dimensiones (curricular, político-ideológica y extensionista) y con acciones definidas en cada una de ellas. Se realiza sobre la base de la unidad entre la educación y la instrucción, teniendo como instrumento básico la orientación elaborada en la Universidad de Matanzas, donde se disponen los criterios para el diagnóstico que se realiza en cada año académico. 

En consecuencia, de la estrategia educativa de la carrera se confeccionan las de los diferentes años académicos, a partir de los criterios dispuestos para este diagnóstico inicial que refleja las necesidades e intereses de los estudiantes y en correspondencia con el objetivo integrador del año. Son elaboradas por los estudiantes conducidos por los profesores guías (PG) y profesores principales de año (PPA). Estas estrategias educativas se ajustan a las necesidades metodológicas del colectivo de año y a las particularidades de los estudiantes, con un adecuado balance entre los componentes que responden a las necesidades e intereses de los estudiantes y los profesores.
 
En las estrategias educativas se logra que los estudiantes cumplan con calidad los objetivos trazados en el año, garantizando la calidad del proceso docente educativo, con acciones concretas, generales e individualizadas, en cada una de las dimensiones dispuestas para ello (curricular, extensionista y sociopolítica). Igualmente se cuentan con criterios de medidas que sirven de base para la evaluación integral de los estudiantes que se realiza al final del curso académico, lo que define el complimiento de la estrategia. 

Los estudiantes son los principales protagonistas y artífices fundamentales de las estrategias educativas, lo que garantiza la efectividad de las mismas y que se parezcan a cada grupo. Se priorizan en ella las tareas de impacto, la vinculación a líneas y proyectos de investigación y las orientaciones de la práctica laboral investigativa en correspondencia con el objetivo integrador del año.

\section{Relación entre los diferentes componentes del proceso docente – educativo en la carrera}

La carrera gestiona su proceso docente-educativo con una evidente y efectiva relación entre los componentes académico, laboral e investigativo, que responden a la flexibilidad, calidad y pertinencia exigida en la formación del profesional que demanda el territorio. Los colectivos pedagógicos desarrollan su trabajo docente - metodológico dirigido a su constante perfeccionamiento, destacándose el papel de la Disciplina Principal Integradora como columna vertebral en la formación de la habilidad profesional, la educación a través de la instrucción, la formación de valores, los métodos de enseñanza, las formas organizativas de la enseñanza, los sistemas de evaluación del aprendizaje, el uso de las tecnologías de la información y las comunicaciones y la aplicación de las estrategias curriculares así como el trabajo con los documentos rectores del Proyecto Social Cubano.

\section{Actividad investigativa laboral de los estudiantes}

La actividad investigativa laboral de los estudiantes se desarrolla con calidad y en correspondencia con las líneas y proyectos de investigación, lo que contribuye a la formación y desarrollo de sus modos y esferas de actuación. Desde la Disciplina Principal Integradora se coordina la práctica laboral investigativa, en correspondencia con las necesidades de las instituciones del territorio y el objetivo integrador del año académico. Se realizan asociadas a las entidades laborales de base y unidades docentes, contando en cada caso con un tutor al frente de su formación.

En los últimos años se ha fortalecido y consolidado el trabajo de la carrera con las unidades docentes y entidades laborales de base. Los directivos de la carrera participan en actividades de dichas entidades y viceversa, en función de lograr un mayor vínculo para la inclusión de estudiantes en las prácticas laborales investigativa y su atención particularizada.

La actividad investigativa laboral, se ha fortalecido con la integración de estudiantes a las sociedades científicas estudiantiles y proyectos de investigación, con estrecha relación con el posgrado. Actualmente el 100\% de los estudiantes están vinculados a proyectos y muestran resultados satisfactorios en la participación en eventos científicos, talleres, sesiones científicas, jornadas científicas estudiantiles.

Una labor importante para el éxito de las prácticas laborales y la investigación lo constituyen las unidades docentes de la carrera, constituidas en entidades laborales de prestigio profesional con los requisitos necesarios para la formación de los modos y esferas de actuación de la profesión. Estas unidades son: el Departamento de Investigación y Desarrollo de la Dirección Provincial de Cultura (Unidad de Desarrollo e Innovación), el CITMA, el CIGET el Centro Provincial de Patrimonio Cultural, el Archivo Histórico Provincial, el Ministerio de Trabajo y Seguridad Social. Algunos profesionales de estas unidades docentes son tutores para la práctica laboral investigativa, las investigaciones y trabajos de diploma, además que son profesores del claustro, lo que demuestra la vinculación de la carrera con profesionales del territorio.

Se muestra satisfacción por parte de los estudiantes con el trabajo realizado por estas instituciones y la atención que le brindan en investigaciones y durante la práctica laboral investigativa.

\section{Estrategias curriculares }

Cada estrategia curricular se trabaja a partir de los modos y esferas de actuación profesional de un gestor sociocultural, lo que aporta conocimientos, valores y herramientas para el ejercicio de la profesión. La aplicación de las mismas se concibe con una visión integradora a partir de las potencialidades que ofrece el Plan de Estudio E, concretándose en el trabajo metodológico de los diferentes años académicos, disciplinas y asignaturas.

Las \textbf{Estrategias Curriculares de la carrera según el Documento del Plan de Estudio E} son:



\underline{\textbf{El uso de la lengua materna}}\\
El uso adecuado de la lengua materna se convierte en un recurso indispensable para poder enfrentar los requerimientos de la especialidad en sus diversos modos y esferas de actuación profesionales, en la medida que no solo garantiza lograr mayor comunicación, comprensión e interpretación en nuestro idioma tanto en sus manifestaciones orales como escritas.

Es por tanto una necesidad formativa transversal y permanente de la carrera atendiendo a la calidad que requiere un profesional de las Ciencias Sociales. La estrategia se aplica en todas las asignaturas de la carrera, potenciando el uso del lenguaje técnico propio del perfil profesional, en las evaluaciones realizadas, lo que hace posible el desarrollo y la interacción del pensamiento, la comunicación, la comprensión y la expresión de los profesionales de la gestión sociocultural en su actuar cotidiano

\underline{\textbf{El uso de la tecnología de la información}}\\
El uso adecuado y pertinente de la tecnología de la información es una exigencia general para todo egresado universitario, lo que se refleja para esta carrera en la concepción de una disciplina de computación en el currículo base, pero también desde asignaturas como Gestión de la Información y el Conocimiento, Metodología de la Investigación y con la optativa Estadística Aplicada a las Ciencias Sociales. Con esta estrategia se alcanzan resultados en la formación profesional de los estudiantes, concretándose en resultados como participación en eventos, uso del correo electrónico, navegación y búsqueda de información en sitios referenciados, trabajo en la plataforma interactiva Moodle, entre otros. Se ha logrado que los estudiantes puedan trabajar con bases de datos remotas y locales, bibliotecas personales digitalizadas, así como el uso de materiales y multimedias. Se evidencia un adecuado nivel de satisfacción de los estudiantes con la orientación que se les ofrece en cada una de las disciplinas y asignaturas para el trabajo con los recursos informáticos. El uso de las tecnologías de la información se considera como aspecto de evaluación sistemático en todas las disciplinas y asignaturas de la carrera.

\underline{\textbf{Estrategia de idioma inglés}}\\
Como resultado del perfeccionamiento de los planes de estudio en la Educación Superior de Cuba, que tiene por objetivo alcanzar la formación integral de los futuros profesionales, se tiene en cuenta la necesidad de que los egresados muestren competencia comunicativa en una lengua extranjera, principalmente en idioma inglés por ser la de más amplia difusión internacional a la luz de las necesidades y proyecciones del desarrollo del país y en consonancia con las tendencias internacionales el dominio de esta lengua se convierte en un imperativo de primer orden. 

En el diseño de los Planes de Estudio E se concibe que la disciplina Idioma Inglés no tenga presencia en el currículo y se considere como requisito de graduación el dominio de este idioma en un nivel pre establecido. Por tanto, esta estrategia se trabaja de manera individualizada, a partir de las necesidades concretas de cada estudiante, a partir de los diagnósticos realizados por los profesores del Centro de Idiomas, que determinan el nivel de cada estudiante y las acciones a seguir.
 
Igualmente, desde la carrera se potencia el trabajo con esta estrategia desde cada disciplina y cada una de las asignaturas, con frases para interpretar, búsquedas bibliográficas especializadas en temáticas específicas, en cada trabajo de curso y diploma se entrega un resumen en idioma inglés, lo que permite ampliar el espectro cultural y profesional del futuro gestor sociocultural, dado la variedad de su perfil profesional.

\underline{\textbf{Estrategia para la educación ambiental}}\\
La educación ambiental se convierte en recurso profesional de la carrera y por ello exige una adecuada y sostenida formación desde las disciplinas y en el desarrollo de la actividad laboral. Desde actividades extensionistas y académicas se promueve la educación ambiental en todos los años de la carrera.

En el currículo se imparten asignaturas como Gestión Medioambiental, de Salud y Prevención Social, Naturaleza y Sociedad, Patrimonio Arqueológico Matancero, Políticas Sociales y Públicas, Estudios Poblacionales, Metodología del Trabajo Social y Comunitario, Historia y Cultura Regional, Gestión Sociocultural del Patrimonio, que direccionan el trabajo con esta estrategia de manera particular. Hay presentaciones en jornadas científicas estudiantiles, actividades concretas orientadas desde la práctica laboral investigativa y trabajos de diplomas de estudiantes con resultados específicos referidos a este tema que forman parte del Observatorio Ambiental COSTATENAS.

Los estudiantes de la carrera desarrollan acciones en sus Consejos Populares para potenciar la educación ambiental y la prevención de salud entre sus habitantes, donde existe círculos de interés con niños para trabajar desde edades tempranas estos temas

Además de las Estrategias Curriculares definidas en el Documento del Plan de Estudio E, la carrera trabaja las siguientes estrategias:

\underline{\textbf{Estrategia de Historia de Cuba}}\\
Se cumplimenta durante todo el desarrollo del programa al Incorporar las nociones y elementos de los sistemas de conocimientos históricos a los contenidos de la asignatura Historia de Cuba según corresponden , al vincular la relación de la situación socio-histórica con el desarrollo de la Política Cultural Cubana y la evolución de los procesos culturales que desemboca en la concepción de potenciar el trabajo en las comunidades mediante los proyectos socioculturales en busca de la transformación individual y social. Utilizar los elementos de la historia local vinculados a las comunidades donde se realiza diagnóstico, y existen varios proyectos socioculturales de relevancia, además de hacer alusión a las fechas históricas, según los días de clases. 

\underline{\textbf{Estrategia de Formación Jurídica}}\\
Se trabaja con intencionalidad en el marco legislativo para la gestión de proyectos en Cuba. Intensificando en el estudio de la Ley No. 44/2012 del CITMA. Se identifican las fuentes de financiamiento para el desarrollo de proyectos culturales y se explica la diversidad de la cooperación internacional para el financiamiento de proyectos socioculturales.

\underline{\textbf{Estrategia de Preparación para la Defensa}}\\
Al propiciar el estudio e investigación de las comunidades y la creación de proyectos socioculturales comunitarios, como una forma de facilitar procesos de transformación individual y social, se propicia el afianzamiento del sentido de pertenencia, el amor al barrio, a los valores patrios y se trabaja por el respeto a las personalidades y la defensa de la identidad y cultura nacionales. También como parte de la estrategia de defensa se trabaja en la prevención de las indisciplinas sociales, adicciones, enfermedades degenerativas y la divulgación de estilos de vida saludables. De esta forma se trabaja desde la asignatura por la continuidad del proyecto social cubano.

\underline{\textbf{Estrategia de Formación Económica y de Dirección}}\\
Se trabaja en la potenciación de las habilidades a lograr con las asignaturas de la Disciplina Principal Integradora, en Gestión Sociocultural, Gestión de Proyectos y Evaluación de Impactos y Gestión Organizacional y de Gobierno de manera particular, aunque en todas las del currículo se contribuye y potencia el trabajo con esta estrategia. La gestión de proyectos socioculturales comunitarios tiene como principio metodológico fundamental el análisis de los procesos socioculturales en su relación y determinación, en última instancia por los procesos económicos. Además, al aplicar esta concepción se profundiza en el conocimiento de los procesos de dirección, en sus diferentes funciones básicas: planeación, organización, dirección y control.\\

\textbf{Fortalezas:}
\begin{itemize}
	\setlength\itemsep{-0.5em}
	\item El PPD garantiza la formación integral de los estudiantes y la apropiación de las esferas y modos de actuación profesional. Se garantiza igualmente la superación posgraduada, con un diseño planificado y coherente que responde al perfil profesional y necesidades del territorio.  
	\item Elevado predominio y liderazgo de profesores con categoría docente Titular y Auxiliar en la coordinación del trabajo metodológico, a nivel de departamento, carrera, colectivos de año y disciplinas.
	\item La estrategia educativa se concibe como un sistema que rige el trabajo metodológico, con un adecuado balance entre sus dimensiones, para cumplir con calidad los objetivos de la formación del profesional y el objetivo integrador de cada año académico
	\item Alto prestigio de las unidades docentes de la carrera, que contribuyen significativamente a la formación de los estudiantes en sus modos y esferas de actuación. En el claustro existen profesores a tiempo parcial que son profesionales que laboran en estas UD y que son fundadores de la carrera, lo que garantiza una excelente relación entre la carrera y su impacto en el territorio matancero, avalado con resultados concretos de investigación. 
	\item Excelente labor de la DPI en la formación curricular e investigativa de los estudiantes, desde la práctica laboral investigativa hasta el ejercicio de culminación de estudios con sus diferentes tipologías (Trabajo de Diploma, Examen Estatal y Portafolio). Para el logro efectivo de la culminación de estudios se tiene en cuenta la ubicación laboral anticipada de cada estudiante y cómo da respuesta a las problemáticas de la institución en particular y del territorio en general.
	
	Tiene gran valía para la carrera la realización cada año de las defensas públicas del plan de estudios, con empleadores, claustro y estudiantes, lo que permite un acercamiento constante y actualización a los intereses del territorio, necesidades de los empleadores y expectativas de los estudiantes. A partir de ellas se perfecciona el currículo propio y optativo / electivo, el cual se gestiona de forma flexible y participativa, respondiendo a los objetivos de la formación del profesional y a las necesidades del territorio.
\end{itemize}


\textbf{Debilidades:}
\begin{itemize}
	\setlength\itemsep{-0.5em}
	\item No se declaran 
\end{itemize}